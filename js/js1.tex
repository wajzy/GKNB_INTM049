\documentclass[usenames,dvipsnames,aspectratio=169]{beamer}
\usepackage{../common/web}

\title[Web technológiák - JavaScript]{Web technológia}
\subtitle{JavaScript}

% JavaScript
\lstdefinelanguage{JavaScript}{
  morekeywords={typeof, new, true, false, catch, function, return, null, undefined, try, catch, finally, throw, var, let, const, in, of, while, do, for, if, else, switch, case, default, break, continue, class, interface, instanceof, debugger, delete, enum, export, extends, implements, import, package, private, protected, public, this, super, static, void, with, yield},
  morecomment=[s]{/*}{*/},
  morecomment=[l]//,
  morestring=[b]",
  morestring=[b]'
}

\begin{document}

%1
\begin{frame}[plain]
  \titlepage
  \logoalul
\end{frame}

%\section{Értékek, típusok, műveletek}

\subsection{Számok}

\begin{frame}
    Jellemzők:
    \begin{itemize}
        \item Egyetlen típus létezik csak: 64 bites lebegőpontos ábrázolás
        \item Pl. \texttt{42}, \texttt{12.34}, \texttt{-34.56}, \texttt{1e3}, \texttt{-1e3}, \texttt{1e-3}, \texttt{-1e-3}, \texttt{-1.23e-4}, \texttt{-1.23E+4}, \dots
        \item Különleges értékek: \texttt{Infinity}, \texttt{-Infinity}, \texttt{NaN}
        \item Pl. \texttt{0/0} \kiemelZ{$\to$ NaN}, \texttt{1/0} \kiemelZ{$\to$ Infinity}
    \end{itemize}
    \vfill
    Operátorok (\hiv{\href{https://developer.mozilla.org/en-US/docs/Web/JavaScript/Reference/Operators/Operator_Precedence\#table}{Precedencia táblázat}})
    \begin{itemize}
        \item[$+$] \texttt{5+3} \kiemelZ{$\to$ 8}
        \item[$-$] \texttt{5-3} \kiemelZ{$\to$ 2} 
        \item[$\times$] \texttt{5*3} \kiemelZ{$\to$ 15}
        \item[$/$] \texttt{5/3} \kiemelZ{$\to$ 1.6666666666666667}
        \item[$\%$] \texttt{5\%3} \kiemelZ{$\to$ 2}, \texttt{-5\%3} \kiemelZ{$\to$ -2}, \texttt{5\%-3} \kiemelZ{$\to$ 2} 
        \item[$**$] \texttt{5**3} \kiemelZ{$\to$ \texttt{125}} 
    \end{itemize}
\end{frame}

\subsection{Karakterláncok}

\begin{frame}
    Jellemzők
    \begin{itemize}
        \item Unicode, 16 bit karakterenként
        \item Nincs specifikus típus egyetlen karakter tárolására
        \item Jelölés: \texttt{'}-ok vagy \texttt{"}-ek között
        \item Pl. \texttt{'JavaScript'}, \texttt{"JavaScript"}, \texttt{"Guns 'n' Roses"}, \texttt{"Egy\textbackslash nKettő\textbackslash nHárom"}, \texttt{'Guns \textbackslash 'n\textbackslash' Roses'}, \texttt{"Új sor \textbackslash\textbackslash n megadásával kérhető."}
        \item \emph{Template literal:} \texttt{`}-ek között, kifejezések kiértékelése
        \item Pl. \texttt{`5 * 3 = \$\{5*3\}`} \kiemelZ{$\to$ \texttt{"5 * 3 = 15"}}
    \end{itemize}
    \vfill
    Operátor
    \begin{itemize}
        \item[$+$] \texttt{"Java" + 'Script'} \kiemelZ{$\to$ \texttt{"JavaScript"}}
    \end{itemize}
\end{frame}

\subsection{Logikai értékek}

\begin{frame}
    Jellemzők
    \begin{itemize}
        \item Értékek: \texttt{true}, \texttt{false}
        \item Pl. \texttt{5 < 3} \kiemelZ{$\to$ \texttt{false}}
    \end{itemize}
    \vfill
    Logikai operátorok
    \begin{itemize}
        \item[és] \texttt{true \&\& false} \kiemelZ{$\to$ \texttt{false}}
        \item[vagy] \texttt{true || false} \kiemelZ{$\to$ \texttt{true}}
        \item[nem] \texttt{!true} \kiemelZ{$\to$ \texttt{false}}
    \end{itemize}
\end{frame}

\begin{frame}
    Relációs operátorok
    \begin{itemize}
        \item \texttt{==}, \texttt{!=}, \texttt{<}, \texttt{<=}, \texttt{>}, \texttt{>=} 
        \item Pl. \texttt{"Bill" != "Gates"} \kiemelZ{$\to$ \texttt{true}}, \texttt{Infinity == Infinity} \kiemelZ{$\to$ \texttt{true}}, \\ \kiemel{de} \texttt{NaN == NaN} \kiemelZ{$\to$ \texttt{false}} (ld. \hiv{\href{https://developer.mozilla.org/en-US/docs/Web/JavaScript/Reference/Global\_Objects/isNaN}{\texttt{isNaN()}}}, \hiv{\href{https://developer.mozilla.org/en-US/docs/Web/JavaScript/Reference/Global\_Objects/isFinite}{isFinite()}})
        \item Karakterláncok összehasonlítása: karakterkódok alapján
    \end{itemize}
\end{frame}

\subsection{Egyebek}

\begin{frame}
    Üres értékek: valaminek a hiányát jelzik
    \begin{itemize}
        \item \texttt{undefined}
        \item \texttt{null}
    \end{itemize}
    \vfill
    Egyoperandusú operátorok
    \begin{itemize}
        \item[típus] \texttt{typeof(5)} \kiemelZ{$\to$ \texttt{"number"}}, \texttt{typeof("Gizi")} \kiemelZ{$\to$ \texttt{"string"}}
        \item[$-$] \texttt{-(5)} \kiemelZ{$\to$ \texttt{-5}}
    \end{itemize}
    \vfill
    Háromoperandusú operátor
    \begin{itemize}
        \item[?:] \texttt{1<2?"kisebb":"nagyobb"} \kiemelZ{$\to$ \texttt{"kisebb"}}
    \end{itemize}
\end{frame}

\subsection{Automatikus típuskonverzió (\emph{Type coercion})}

\begin{frame}
    Néhány példa:
    \begin{itemize}
        \item \texttt{5 * null} \kiemelZ{$\to$ \texttt{0}}
        \item \texttt{"5" - 3} \kiemelZ{$\to$ \texttt{2}}
        \item \texttt{"5" + 3} \kiemelZ{$\to$ \texttt{"53"}}
        \item \texttt{"öt" * 3} \kiemelZ{$\to$ \texttt{NaN}}, \texttt{5 * undefined} \kiemelZ{$\to$ \texttt{NaN}}
        \item \texttt{false == 0} \kiemelZ{$\to$ \texttt{true}}, \texttt{true == 1} \kiemelZ{$\to$ \texttt{true}}, \texttt{true == 2} \kiemelZ{$\to$ \texttt{false}}, \\ \texttt{"" == false} \kiemelZ{$\to$ \texttt{true}}
        \item Definiált az érték? \texttt{null == undefined} \kiemelZ{$\to$ \texttt{true}}, \texttt{null == 0} \kiemelZ{$\to$ \texttt{false}}
    \end{itemize}
    \vfill
    Típusok egyezését megkövetelő operátorok: \texttt{===}, \texttt{!==}
\end{frame}

%\section{Változók, konstansok, vezérlési szerkezetek}

\subsection{Változók és konstansok}

\begin{frame}
    Változók (\emph{variable}, \emph{binding})
    \begin{itemize}
        \item Deklaráció: \texttt{let} (blokk hatáskör), \texttt{var} (függvény hatáskör)
        \item Inkább tekinthető értékre mutató referenciának, mint valódi tárolónak
    \end{itemize}
    \begin{exampleblock}{Példa}
        a \kiemelZ{$\to$ ReferenceError: a is not defined}\\
        let a\\
        a \kiemelZ{$\to$ undefined}\\
        a = 5\\
        a \kiemelZ{$\to$ 5}\\
        let b = 3, c\\
        a * b \kiemelZ{$\to$ 15}\\
    \end{exampleblock}
\end{frame}

\begin{frame}
    Konstansok
    \begin{itemize}
        \item \texttt{const}
    \end{itemize}
    \begin{exampleblock}{Példa}
        const c = 3.14\\
        c = 2 \kiemelZ{$\to$ TypeError: invalid assignment to const 'c'}
    \end{exampleblock}
    \vfill
    Névadási szabályok
    \begin{itemize}
        \item betűket, számokat, \$ és \_ karaktereket tartalmazhat
        \item számjeggyel nem kezdődhet
        \item nem lehet foglalt szó (pl. \texttt{let})
        \item kis- és nagybetűket megkülönbözteti
        \item javasolt stílus: \emph{camel case} (\texttt{hosszuValtozoNeve})
    \end{itemize}
\end{frame}

\begin{frame}
    Változókkal használható (összetett és unáris) operátorok
    \begin{itemize}
        \item \texttt{+=}, \texttt{-=}, \texttt{*=}, \texttt{/=}, \texttt{\%=}, \texttt{\&\&=}, \texttt{||=}, \texttt{**=}, \dots 
        \item \texttt{++}, \texttt{-{-}}
    \end{itemize}
    \vfill
    Környezet (\emph{environment})
    \begin{itemize}
        \item adott pillanatban létező változók és értékeik
        \item gyakorlatilag soha nincs üres környezet
    \end{itemize}
    \vfill
    Megjegyzések
    \begin{itemize}
        \item // egysoros
        \item /* több \\ \quad soros */
    \end{itemize}
\end{frame}

\subsection{Vezérlési szerkezetek}

\begin{frame}
    Szelekció
    \begin{columns}[T]
        \column{0.5\textwidth}
        \begin{itemize}
          \item \texttt{if}(\emph{feltétel}) \emph{utasítás};
          \item \texttt{if}(\emph{feltétel}) \{\\
          \qquad // utasítások \\
          \}
          \item \texttt{if}(\emph{feltétel}) \{ \\
            \qquad // igaz ág utasításai \\
            \} \texttt{else} \{ \\
            \qquad // hamis ág utasításai \\
            \}
        \end{itemize}
        \column{0.5\textwidth}
        \begin{itemize}
          \item Mikor \kiemel{nem} teljesül a \emph{feltétel}?
          \begin{itemize}
            \item false
            \item 0
            \item ""
            \item NaN
            \item null
            \item undefined
          \end{itemize}
        \end{itemize}
      \end{columns}
\end{frame}

\begin{frame}
    Több irányú elágazás
    \begin{columns}[T]
        \column{0.4\textwidth}
        \texttt{switch}(\emph{kifejezés}) \{ \\
        \qquad \texttt{case} \emph{érték1}: \\
        \qquad \qquad // utasítások \\
        \qquad \qquad \texttt{break}; \\
        \qquad \texttt{case} \emph{érték2}: \\
        \qquad \texttt{case} \emph{érték3}: \\
        \qquad \qquad // utasítások \\
        \qquad \qquad \texttt{break}; \\
        \qquad \texttt{default}: \\
        \qquad \qquad // utasítások \\
        \qquad \qquad \texttt{break}; \\
        \}
      \column{0.5\textwidth}
        Az értéknek \emph{és} a típusnak is egyeznie kell! \\
        A \emph{default} ág elhagyható.
    \end{columns}
\end{frame}

\begin{frame}
    Ciklusok
    \vfill
    \texttt{for}(\emph{előkészítés}; \emph{ismétlési\_feltétel}; \emph{frissítés}) \{ \\
    \qquad // Ciklusmag utasításai \\
    \}
    \vfill
    \texttt{while}(\emph{ismétlési\_feltétel}) \{ \\
    \qquad // Ciklusmag utasításai \\
    \}
    \vfill
    \texttt{do} \{ \\
    \qquad // Ciklusmag utasításai \\
    \} \texttt{while} (\emph{ismétlési\_feltétel});
    \vfill
    \texttt{break}, \texttt{continue}
\end{frame}

\subsection{Feladatok}

\begin{frame}[fragile]
    \begin{exampleblock}{Háromszög rajzolás (\textattachfile{haromszog.js}{megoldás})}
        A böngésző JavaScript konzolján egy sornyi szöveget a \texttt{console.log()} hívással tud megjeleníteni. Használja ezt a következő háromszög megrajzolására:\\
        \begin{verbatim}
*
**
***
****
*****                
\end{verbatim}
    \end{exampleblock}
\end{frame}

\begin{frame}[fragile]
    \begin{exampleblock}{X rajzolás (\textattachfile{x.js}{megoldás})}
        Most rajzoljon 5x5-ös méretű X-et csillagokból:\\
        \begin{verbatim}
*   *
 * *
  *
 * *
*   *            
\end{verbatim}
    \end{exampleblock}
\end{frame}

\begin{frame}[fragile]
    \begin{exampleblock}{Sakktábla (\textattachfile{sakk.js}{megoldás})}
        Rajzoljon meg egy 8x8-as méretű sakktáblát, szintén csillagokból!\\
        \begin{verbatim}
 * * * *
* * * *
 * * * *
* * * *
 * * * *
* * * *
 * * * *
* * * *
\end{verbatim}
    \end{exampleblock}
\end{frame}

\begin{frame}[fragile]
    \begin{exampleblock}{FizzBuzz (\textattachfile{fizzbuzz.js}{megoldás})}
        Vizsgálja meg az egész számokat 1-től 100-ig, majd a vizsgálat eredményét jelenítse meg egymás alatti sorokban! Ha a szám osztható 3-mal, írja ki, hogy \emph{Fizz}, ha 5-tel osztható, akkor azt, hogy \emph{Buzz}, ha pedig 3-mal és 5-tel is osztható, akkor azt, hogy \emph{FizzBuzz}! Ha egyik számmal sem osztható, akkor írja ki a vizsgált számot!\\
        \begin{verbatim}
1
2
Fizz
4
Buzz
Fizz
...
\end{verbatim}
    \end{exampleblock}
\end{frame}

%\section{Függvények}

\subsection{Függvények létrehozása}

\begin{frame}
    \begin{exampleblock}{Definíció: a függvény, mint \emph{érték} jelenik meg (\textattachfile{hatvanyDef.js}{hatvanyDef.js})}
        \small
        \lstinputlisting[language=JavaScript,numbers=left]{hatvanyDef.js}
    \end{exampleblock}
\end{frame}

\begin{frame}
    \begin{exampleblock}{Deklaráció: helye a hatókörön belül bárhol lehet (\textattachfile{hatvanyDek.js}{hatvanyDek.js})}
        \small
        \lstinputlisting[language=JavaScript,numbers=left]{hatvanyDek.js}
    \end{exampleblock}
\end{frame}

\begin{frame}
    \begin{exampleblock}{Nyíl (\emph{arrow}) függvény: tömörebb megadás (\textattachfile{hatvanyNyil.js}{hatvanyNyil.js})}
        \small
        \lstinputlisting[language=JavaScript,numbers=left]{hatvanyNyil.js}
    \end{exampleblock}
\end{frame}

\begin{frame}
    Nyíl függvények
    \begin{itemize}
        \item Ha pontosan egy paramétert fogad, a paraméterlista körüli zárójelek elhagyhatóak
        \item Ha egyetlen paramétert sem fogad, üres zárójelpár jelzi a paraméterlistát
        \item Ha a függvény teste egyetlen kifejezés értékét szolgáltatja, a \texttt{return} és a blokk elhagyható
    \end{itemize}
    \begin{exampleblock}{\textattachfile{nyilValtozatok.js}{nyilValtozatok.js}}
        \small
        \lstinputlisting[language=JavaScript,numbers=left]{nyilValtozatok.js}
    \end{exampleblock}
\end{frame}

\subsection{Hatókör, \emph{lexical scope}}

\begin{frame}
    \begin{exampleblock}{\textattachfile{hatokor.js}{hatokor.js}}
        \scriptsize
        \lstinputlisting[language=JavaScript,numbers=left]{hatokor.js}
    \end{exampleblock}
\end{frame}

\subsection{Függvények paraméterezése}

\begin{frame}
    Paraméterezés
    \begin{itemize}
        \item Nem ellenőrzi híváskor sem a paraméterek számát, sem azok típusát! $\to$ felesleges paramétereket figyelmen kívül hagyja, a hiányzók értéke \texttt{undefined}
        \item A \texttt{return} nélküli, vagy a \texttt{return} után kifejezést nem tartalmazó függvények visszatérési értéke \texttt{undefined}
        \item Tetszőleges számú paramétert fogadó fv. is készíthető (ld. később)
    \end{itemize}
    \begin{exampleblock}{\textattachfile{parameter1.js}{parameter1.js}}
        \small
        \vspace{-0.3cm}
        \lstinputlisting[language=JavaScript,numbers=left]{parameter1.js}
        \vspace{-0.3cm}
    \end{exampleblock}
\end{frame}

\begin{frame}
    \begin{exampleblock}{\textattachfile{parameter2.js}{parameter2.js}}
        \lstinputlisting[language=JavaScript,numbers=left]{parameter2.js}
    \end{exampleblock}
\end{frame}

\begin{frame}
    \begin{exampleblock}{Régi módszer hiányzó paraméterek kezelésére (\textattachfile{hatvanyAlapertelmezettRegi.js}{hatvanyAlapertelmezettRegi.js})}
        \small
        \lstinputlisting[language=JavaScript,numbers=left]{hatvanyAlapertelmezettRegi.js}
    \end{exampleblock}
\end{frame}

\begin{frame}
    \begin{exampleblock}{Alapértelmezett paraméter érték (\textattachfile{hatvanyAlapertelmezett.js}{hatvanyAlapertelmezett.js})}
        \small
        \lstinputlisting[language=JavaScript,numbers=left]{hatvanyAlapertelmezett.js}
    \end{exampleblock}
\end{frame}

\begin{frame}
    Tulajdonságok:
    \begin{itemize}
        \item Paraméterek átadása balról jobbra, akár az alapértelmezett értékek felülírásával is
        \item Alapértelmezett érték kiszámítható kifejezéssel, akár fv. hívással is
        \item Ezek minden egyes híváskor kiértékelődnek
        \item Minden, a paramétertől balra lévő további paraméter használható inicializálásra
        \item \hiv{\href{https://developer.mozilla.org/en-US/docs/Web/JavaScript/Reference/Functions/Default\_parameters?retiredLocale=hu}{További részletek}}
    \end{itemize}
\end{frame}

\subsection{Függvények, mint értékek}

\begin{frame}
    A függvények \emph{értékek}:
    \begin{itemize}
        \item függvények átadhatók más függvénynek paraméterként,
        \item függvény visszatérési értéke lehet függvény,
        \item függvény beágyazható másik függvénybe.
    \end{itemize}
    \emph{Magasabbrendű függvények} (Higher-Order Functions, HOF): olyan függvények, melyek fv. paramétert fogadnak, vagy fv.-t adnak vissza
    \begin{exampleblock}{\textattachfile{paramFv1.js}{paramFv1.js}}
        \small
        \lstinputlisting[language=JavaScript,numbers=left]{paramFv1.js}
    \end{exampleblock}
\end{frame}

\begin{frame}
    \begin{exampleblock}{Névtelen (\emph{anonymous}) függvények (\textattachfile{paramFv2.js}{paramFv2.js})}
        \lstinputlisting[language=JavaScript,numbers=left]{paramFv2.js}
    \end{exampleblock}
\end{frame}

\begin{frame}
    \begin{exampleblock}{Függvények definiálása és azonnali hívása (\textattachfile{paramFv3.js}{paramFv3.js})}
        \lstinputlisting[language=JavaScript,numbers=left]{paramFv3.js}
    \end{exampleblock}
\end{frame}

\begin{frame}
    Zárványok (\emph{closure})
    \begin{itemize}
        \item<1-> Mi történik, ha egy \emph{külső} függvény \emph{lokális} változóit eléri egy \emph{belső} függvény, amit \emph{meghívunk azután, hogy} az őt létrehozó \emph{külső függvényből kiléptünk}?
        \item<2-> A függvény megőrzi futtatási környezetét
    \end{itemize}
\end{frame}

\begin{frame}
    \begin{exampleblock}{\textattachfile{zarvany.js}{Currying:} egy több paramétert fogadó függvény megvalósítása kevesebb paramétert fogadó magasabb rendű függvényekkel}
        \small
        \vspace{-.3cm}
        \lstinputlisting[language=JavaScript,numbers=left]{zarvany.js}
        \vspace{-.3cm}
    \end{exampleblock}
\end{frame}

\subsection{Rekurzió}

\begin{frame}
    \begin{exampleblock}{Rekurzív hatványozás (\textattachfile{rekurzio.js}{rekurzio.js})}
        \small
        \lstinputlisting[language=JavaScript,numbers=left]{rekurzio.js}
    \end{exampleblock}
\end{frame}

\subsection{Feladatok}

\begin{frame}
    \begin{exampleblock}{Fibonacci-számok (\textattachfile{fibonacci.js}{fibonacci.js})}
        Fibonacci-sorozat: másodrendben rekurzív sorozat. Képzeletbeli nyúlcsalád növekedése: hány pár nyúl lesz $n$ hónap múlva, ha
        \begin{itemize}
            \item az első hónapban csak egyetlen újszülött nyúl-pár van,
            \item az újszülött nyúl-párok két hónap alatt válnak termékennyé,
            \item minden termékeny nyúl-pár minden hónapban egy újabb párt szül,
            \item és a nyulak örökké élnek.
        \end{itemize}
        \vfill
        $F_n = \left\{ \begin{array}{ll}
            0, & \textrm{ha $n=0$}\\
            1, & \textrm{ha $n=1$}\\
            F_{n-1} + F_{n-2} & \textrm{ha $n>1$}
        \end{array} \right.$ \\
        Készítse el azt a \texttt{fibonacci} függvényt, melynek paramétere a sorozat valamely elemének indexe (\emph{n}), visszatérési értéke a sorozat megfelelő eleme!
    \end{exampleblock}
\end{frame}

\begin{frame}
    \begin{exampleblock}{Négyzetgyökvonás (\textattachfile{gyok.js}{gyok.js})}
        \small
        Készítse el a \texttt{gyok} függvényt, mely Newton módszerrel meghatározza és visszatérési értékként szolgáltatja paraméterének négyzetgyökét! \\
        A módszer iteratív: egy sorozat egymást követő tagjait kell kiszámolni, melyek általában nagyon gyorsan konvergálnak a keresett eredményhez. A sorozat első elemét célszerű lenne a megoldás közeléből választani, de az egyszerűség kedvéért legyen ez nálunk mindig 10. Ha az utolsóként meghatározott tag értéke \( 10^{-6} \)-nál nem nagyobb mértékben tér el az utolsó előttiként kiszámolttól, akkor ezt az utolsóként kiszámolt értéket tekintjük a megoldásnak. A Newton módszer szerint a sorozat tagjait általánosan a következőképpen határozzuk meg: \( x_{n+1} = x_n - \frac{f(x_n)}{f'(x_n)} \) \\
        Konkrétan a négyzetgyökvonás esetén, ha pl. az \( x^2 = 612 \) (itt 612 a \texttt{gyok} függvény aktuális paraméterének feleltethető meg) zérushelyét keressük, azaz \( f(x) = x^2 - 612 \) akkor \( f'(x) = 2x \). \\
        Ebből adódik, hogy \( x_1 = x_0 - \frac{f(x_0)}{f'(x_0)} = 10 - \frac{10^2 - 612}{2 \cdot 10} = 35.6 \) majd \( x_2 = x_1 - \frac{f(x_1)}{f'(x_1)} = 35.6 - \frac{35.6^2 - 612}{2 \cdot 35.6} = 26.3955056 \), stb.
    \end{exampleblock}
\end{frame}

\begin{frame}
    \begin{exampleblock}{Szinusz függvény (\textattachfile{sin.js}{sin.js})}
        Írja meg azt a \texttt{sin} függvényt, amely visszaadja a paraméterként kapott, radiánban mért szög szinuszát!\\
        \smallskip
        A keresett érték meghatározható a szinusz függvény sorba fejtésével: \( sin(x) = \sum_{n=0}^\infty \frac{(-1)^n}{(2n+1)!} x^{2n+1} \) azaz \( sin(x) = x-\frac{x^3}{3!}+\frac{x^5}{5!}-\frac{x^7}{7!}+\dots \)\\
        \smallskip
        A függvénynek természetesen nem kell végtelen sok tagot, illetve azok összegét meghatároznia. Elegendő, ha a függvény \( \epsilon = 10^{-6} \) pontossággal kiszámítja az eredményt.
    \end{exampleblock}
    \vfill
    \begin{exampleblock}{Legnagyobb közös osztó (\textattachfile{lnko.js}{lnko.js})}
        Valósítsa meg az \hiv{\href{https://hu.wikipedia.org/wiki/Euklideszi\_algoritmus}{Euklideszi algoritmust}} két egész szám legnagyobb közös osztójának meghatározásához!
    \end{exampleblock}
\end{frame}

\section{Objektumokról részletesebben}

\subsection{\texttt{this}}

\begin{frame}
    Objektumok: 
    \begin{itemize}
        \item egységbe zárt (encapsulation) adattagok és metódusok (függvények)
        \item a rejtett \texttt{this} kötés kapcsolja össze az objektum tulajdonságait
        \item minden tulajdonság nyilvános
    \end{itemize}
    \begin{exampleblock}{\textattachfile{teglalap.js}{Objektum literál}}
        \small
        \lstinputlisting[language=JavaScript,numbers=left,linerange=1-7]{teglalap.js}
    \end{exampleblock}
\end{frame}

\begin{frame}
    \begin{exampleblock}{\textattachfile{teglalap.js}{Metódusok}}
        \scriptsize
        \lstinputlisting[language=JavaScript,numbers=left,firstnumber=8,linerange=8-19]{teglalap.js}
    \end{exampleblock}
\end{frame}

\begin{frame}
    \begin{exampleblock}{\textattachfile{teglalap.js}{A \texttt{call} metódus}}
        \scriptsize
        \lstinputlisting[language=JavaScript,numbers=left,firstnumber=21,linerange=21-28]{teglalap.js}
    \end{exampleblock}
\end{frame}

\begin{frame}
    \begin{exampleblock}{\textattachfile{teglalap.js}{Nyíl függvények és \texttt{function} közti különbségek}}
        \small
        \lstinputlisting[language=JavaScript,numbers=left,firstnumber=30,linerange=30-41]{teglalap.js}
    \end{exampleblock}
\end{frame}

\subsection{Prototípusok}

\begin{frame}
    JavaScriptben nincsenek osztályok: az objektumok más objektumok (prototípusok) alapján jönnek létre, azt kiegészítve.\\
    A prototípusnak is lehet prototípusa: származtatási hierarchia, fa struktúra, csúcsán: \texttt{Object.prototype}; ez a közös ős.
    \begin{exampleblock}{\textattachfile{prototipus.js}{Prototípusok felderítése}}
        \scriptsize
        \lstinputlisting[language=JavaScript,numbers=left,firstnumber=2, linerange=2-5]{prototipus.js}
    \end{exampleblock}
\end{frame}

\begin{frame}
    \begin{exampleblock}{\textattachfile{prototipus.js}{Objektum létrehozása alapértelmezett prototípussal}}
        \lstinputlisting[language=JavaScript,numbers=left,firstnumber=7,linerange=7-16]{prototipus.js}
    \end{exampleblock}
\end{frame}

\begin{frame}
    \scriptsize
    \begin{exampleblock}{\textattachfile{prototipus.js}{Objektum létrehozása adott prototípussal}}
        \scriptsize
        \vspace{-.3cm}
        \lstinputlisting[language=JavaScript,numbers=left,firstnumber=18,linerange=18-35]{prototipus.js}
        \vspace{-.3cm}
    \end{exampleblock}
\end{frame}

\subsection{Konstruktorok}

\begin{frame}
    \footnotesize
    Prototípusnak szánt objektumba jellemzően csak olyan tulajdonságot tesznek, melynek értékét minden ebből 
    származó objektumnak tartalmaznia kell. \\
    Konstruktor: új objektum létrehozása adott prototípusból,
    az egyedre jellemző értékek hozzáadásával.
    \begin{exampleblock}{\textattachfile{konstruktor1.js}{Objektum létrehozása konstruktorral}}
        \scriptsize
        \vspace{-.2cm}
        \lstinputlisting[language=JavaScript,numbers=left,linerange=1-14]{konstruktor1.js}
        \vspace{-.2cm}
    \end{exampleblock}
\end{frame}

\begin{frame}
    \begin{exampleblock}{\textattachfile{konstruktor1.js}{Objektum létrehozása konstruktorral}}
        \small
        \lstinputlisting[language=JavaScript,numbers=left,firstnumber=16,linerange=16-23]{konstruktor1.js}
    \end{exampleblock}
\end{frame}

\begin{frame}
    \emph{Majdnem} ugyanez történik, ha a \texttt{new} kulcsszót írjuk egy függvény elé, azaz konstruktorként fog viselkedni.
    \begin{exampleblock}{\textattachfile{konstruktor2.js}{Objektum létrehozása \texttt{new} kulcsszóval}}
        \footnotesize
        \lstinputlisting[language=JavaScript,numbers=left,linerange=1-11]{konstruktor2.js}
    \end{exampleblock}
\end{frame}

\begin{frame}
    \begin{exampleblock}{\textattachfile{konstruktor2.js}{Objektum létrehozása \texttt{new} kulcsszóval}}
        \footnotesize
        \lstinputlisting[language=JavaScript,numbers=left,firstnumber=12,linerange=12-19]{konstruktor2.js}
    \end{exampleblock}
    Minden függvénynek van \texttt{prototype} tulajdonsága, a konstruktornak is. Ez egy lecserélhető üres objektum, 
    de a meglévőhöz is hozzáadhatók új tulajdonságok, mint a példában. 
\end{frame}

\subsection{\texttt{class} és \texttt{constructor}}

\begin{frame}
    2015-től lehet használni a \texttt{class} és \texttt{constructor} kulcsszavakat. A háttérben ettől még 
    ugyanaz történik (syntactic sugar), továbbra sem léteznek valódi osztályok a nyelvben. Konstans adatokat nem lehet 
    az objektumhoz adni.
    \begin{exampleblock}{\textattachfile{class.js}{\texttt{class}, \texttt{constructor}}}
        \footnotesize
        \vspace{-.2cm}
        \lstinputlisting[language=JavaScript,numbers=left,linerange=1-11]{class.js}
        \vspace{-.2cm}
    \end{exampleblock}
\end{frame}

\begin{frame}
    \begin{exampleblock}{\textattachfile{class.js}{\texttt{class}, \texttt{constructor}}}
        \footnotesize
        \lstinputlisting[language=JavaScript,numbers=left,firstnumber=13,linerange=13-22]{class.js}
    \end{exampleblock}
\end{frame}

\begin{frame}
    A \texttt{class} kifejezésben is szerepelhet, nem csak utasításban.
    \begin{exampleblock}{\textattachfile{class.js}{\texttt{class} kifejezés}}
        \small
        \lstinputlisting[language=JavaScript,numbers=left,firstnumber=24,linerange=24-33]{class.js}
    \end{exampleblock}
\end{frame}

\begin{frame}
    Az \texttt{Object}-től örökölt metódusok egyedileg felüldefiniálhatók.
    \begin{exampleblock}{\textattachfile{class.js}{Felüldefiniálás}}
        \small
        \lstinputlisting[language=JavaScript,numbers=left,firstnumber=35,linerange=35-45]{class.js}
    \end{exampleblock}
\end{frame}

\subsection{Objektumok használata asszociatív tömbként}

\begin{frame}
    \footnotesize
    Objektumokat használni asszociatív tömbként nem biztonságos:
    \begin{enumerate}
        \item öröklött tulajdonságok is megjelennek kulcsként
        \item implicit típuskonverzió miatt kulcsnak látszódhat egy érték, ami nem az
    \end{enumerate}
    \begin{exampleblock}{\textattachfile{asszociativ.js}{Asszociatív tömb objektummal}}
        \vspace{-.2cm}
        \scriptsize
        \lstinputlisting[language=JavaScript,numbers=left,linerange=1-13]{asszociativ.js}
        \vspace{-.2cm}
    \end{exampleblock}
\end{frame}

\begin{frame}
    Az első problémát megoldja, ha nincs prototípusa az objektumnak.
    \begin{exampleblock}{\textattachfile{asszociativ.js}{Asszociatív tömb objektummal}}
        \scriptsize
        \lstinputlisting[language=JavaScript,numbers=left,firstnumber=15,linerange=15-21]{asszociativ.js}
    \end{exampleblock}
\end{frame}

\begin{frame}
    A másodikon viszont nem segít, sőt mellékhatása is lehet (pl. nyomkövetésnél)
    Megoldás: \texttt{Map} (és \texttt{Set}), ld. később
    \begin{exampleblock}{\textattachfile{asszociativ.js}{Asszociatív tömb objektummal}}
        \lstinputlisting[language=JavaScript,numbers=left,firstnumber=23,linerange=23-27]{asszociativ.js}
    \end{exampleblock}
\end{frame}

\subsection{Szimbólumok}

\begin{frame}
    Az objektumok kulcsaiként eddig mindig \texttt{String}-ek álltak. Ez azonban akkor is lehetővé 
    teszi pl. egy metódus felüldefiniálását vagy lecserélését, ha nem szándékoztuk.
    \begin{exampleblock}{\textattachfile{szimbolumok.js}{Tulajdonságok karakterlánccal megadva}}
        \small
        \vspace{-.2cm}
        \lstinputlisting[language=JavaScript,numbers=left,linerange=1-11]{szimbolumok.js}
        \vspace{-.2cm}
    \end{exampleblock}
\end{frame}

\begin{frame}
    Ezzel szemben a létrehozott szimbólumok mindig egyediek.
    \begin{exampleblock}{\textattachfile{szimbolumok.js}{Tulajdonságok szimbólummal megadva}}
        \scriptsize
        \lstinputlisting[language=JavaScript,numbers=left,firstnumber=13,linerange=13-26]{szimbolumok.js}
    \end{exampleblock}
\end{frame}

\subsection{Iterátorok}

\begin{frame}
    \footnotesize
    A \texttt{for/of} ciklusokkal azok a gyűjtemények járhatók be, melyek megvalósítják az \texttt{iterator} interfészt,
    azaz rendelkeznek egy \texttt{Symbol.iterator} nevű metódussal, ami visszadja a tényleges bejárást biztosító
    iterátor objektumot.
    
    Ennek \texttt{next()} metódusával lehet elkérni a \texttt{value} és \texttt{done} kulcsokkal rendelkező objektumokat. 
    Az első adattag a tényleges, soron következő értéket adja meg, a másodikból kiderül, hogy elérhető-e még egyáltalán további adat.

    Legismertebb megvalósítások a nyelvben: \texttt{Array}, \texttt{String}.
    \begin{exampleblock}{\textattachfile{iterator.js}{Beépített objektumok iterátorral}}
        \scriptsize
        \vspace{-.2cm}
        \lstinputlisting[language=JavaScript,numbers=left,linerange=1-10]{iterator.js}
        \vspace{-.2cm}
    \end{exampleblock}
\end{frame}

\begin{frame}
    Készítsünk a \hiv{\href{https://docs.python.org/3/library/functions.html\#func-range}{Python}} és 
    a \hiv{\href{https://www.php.net/manual/en/function.range.php}{PHP}} mintájára olyan \texttt{Intervallum}
    objektumot, ami a [\texttt{tol}, \texttt{ig}) intervallumból ad vissza egymástól \texttt{lepes}-nyire lévő értékeket!
    \begin{exampleblock}{\textattachfile{iterator.js}{Saját objektum iterátorral}}
        \small
        \lstinputlisting[language=JavaScript,numbers=left,firstnumber=12,linerange=12-21]{iterator.js}
    \end{exampleblock}
\end{frame}

\begin{frame}
    \footnotesize
    \begin{exampleblock}{\textattachfile{iterator.js}{Saját objektum iterátorral}}
        \scriptsize
        \vspace{-.2cm}
        \lstinputlisting[language=JavaScript,numbers=left,firstnumber=23,linerange=23-38]{iterator.js}
        \vspace{-.2cm}
    \end{exampleblock}
\end{frame}

\begin{frame}
    \begin{exampleblock}{\textattachfile{iterator.js}{Saját objektum iterátorral}}
        \lstinputlisting[language=JavaScript,numbers=left,firstnumber=40,linerange=40-45]{iterator.js}
    \end{exampleblock}
\end{frame}

\subsection{getter, setter, static}

\begin{frame}
    JS-ben valamennyi adattag nyilvánosan elérhető. Néha viszont nem szeretnénk (redundáns) adattagokat létrehozni, abban folyamatosan
    adatokat tárolni és azt frissíteni, amikor az objektum változik. Ilyenkor létrehozhatunk olyan metódusokat, melyek adattagnak
    tűnnek a külvilág számára, és a számított adatokat szolgáltatják, amikor szükség van rájuk.
    \vfill
    Hasonlóképpen létrehozhatunk adattagnak tűnő metódusokat tárolt értékek beállítására, input ellenőrzésére is.
    \vfill
    Időnként egy adat nem példányhoz kötődik, hanem az ,,osztályhoz'' (ld. \texttt{Math.PI}). Ezeket célszerű csak egyszer tárolni
    $\to$ statikus adattag. Hasonlóan, ha egy metódus nem dolgozik a példány adataival, megjelölhetjük statikusként (pl. \texttt{Math.sin()}).
    Hívásakor a prototípus/osztály nevével minősítjük. Jellemzően objektumok létrehozására, másolására használják őket.
\end{frame}

\begin{frame}
    \begin{exampleblock}{\textattachfile{gallon.js}{getter, setter, static}}
        \scriptsize
        \lstinputlisting[language=JavaScript,numbers=left,linerange=1-15]{gallon.js}
    \end{exampleblock}
\end{frame}

\begin{frame}
    \begin{exampleblock}{\textattachfile{gallon.js}{getter, setter, static}}
        \footnotesize
        \lstinputlisting[language=JavaScript,numbers=left,firstnumber=17,linerange=17-24]{gallon.js}
    \end{exampleblock}
\end{frame}

\subsection{Származtatás}

\begin{frame}
    Származtatás megvalósításához, az ős megnevezésére megjelent az \texttt{extends} kulcsszó. Az ősre \texttt{super} segítségével lehet
    hivatkozni.
    \begin{exampleblock}{\textattachfile{szarmaztatas.js}{Származtatás}}
        \scriptsize
        \vspace{-.2cm}
        \lstinputlisting[language=JavaScript,numbers=left,linerange=1-14]{szarmaztatas.js}
        \vspace{-.2cm}
    \end{exampleblock}
\end{frame}

\begin{frame}
    \begin{exampleblock}{\textattachfile{szarmaztatas.js}{Származtatás}}
        \scriptsize
        \vspace{-.2cm}
        \lstinputlisting[language=JavaScript,numbers=left,firstnumber=16,linerange=16-31]{szarmaztatas.js}
        \vspace{-.2cm}
    \end{exampleblock}
\end{frame}

\begin{frame}
    \begin{exampleblock}{\textattachfile{szarmaztatas.js}{Származtatás}}
        \small
        \lstinputlisting[language=JavaScript,numbers=left,firstnumber=33,linerange=33-39]{szarmaztatas.js}
    \end{exampleblock}
    Az \texttt{instanceof} operátorral ellenőrizhető egy objektum (akár közvetett) prototípusa.
\end{frame}

\subsection{Feladatok}

\begin{frame}
    \begin{exampleblock}{Lekerekített téglalap (\textattachfile{lekerekitett.js}{lekerekitett.js})}
        Készítsen \texttt{Teglalap} ,,osztályt'', melynek konstruktora megkapja paraméterként a síkidom szélességét, magasságát,
        és egy logikai értéket, melyből kiderül, hogy rajzolásnál csak a körvonalat kell megrajzolni, vagy a téglalap belsejét is
        ki kell tölteni * karakterekkel! A \texttt{rajz()} metódus adja vissza a rajzot egy \texttt{String} formájában!

        Származtasson ebből egy \texttt{Lekerekitett} nevű osztályt, amelynek konstruktora egy további paramétert fogad, a sarkok
        lekerekítési sugarát!
    \end{exampleblock}
\end{frame}

\end{document}

% objektum literálok tulajdonságai, létrehozás: tulajdonságok, metódusok (null, undefined: nincsenek tulajdonságaik)
% tulajdonságok utólagos hozzáadása, elvétele (delete), tartalmazás (in)
% ha csak a változót betesszük az objektumba, a változó neve lesz a kulcs, az értéke az érték
% operátorok, for in/of ciklusok
% az Object objektum: keys, assign
% tömbök kezelése
% változó számú paraméterlistájú fv.-ek, arguments, rest parameter, destructuring
% kapcsolat tömbök és objektumok között, indexelés vs. tulajdonság kiválasztás
% többi alaptípus, typeof eredménye, metódusok, immutable
% console.log((4).toString().padStart(3, "0")); String.repeat
% Math, JSON
