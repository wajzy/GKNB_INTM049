\documentclass[usenames,dvipsnames,aspectratio=169]{beamer}
\usepackage{../common/web}

\title[Web technológiák - JavaScript]{Web-technológia}
\subtitle{JavaScript}

\begin{document}

%1
\begin{frame}[plain]
  \titlepage
  \logoalul
\end{frame}

\section{Értékek (literálok), típusok, műveletek}

\subsection{Számok}

\begin{frame}
    Jellemzők:
    \begin{itemize}
        \item Egyetlen típus létezik csak: 64 bites lebegőpontos ábrázolás
        \item Pl. \texttt{42}, \texttt{12.34}, \texttt{-34.56}, \texttt{1e3}, \texttt{-1e3}, \texttt{1e-3}, \texttt{-1e-3}, \texttt{-1.23e-4}, \texttt{-1.23E+4}, \dots
        \item Különleges értékek: \texttt{Infinity}, \texttt{-Infinity}, \texttt{NaN}
        \item Pl. \texttt{0/0} $\to$ NaN, \texttt{1/0} $\to$ Infinity
    \end{itemize}
    \vfill
    Operátorok
    \begin{itemize}
        \item[$+$] \texttt{5+3} $\to$ 8
        \item[$-$] \texttt{5-3} $\to$ 2 
        \item[$\times$] \texttt{5*3} $\to$ 15
        \item[$/$] \texttt{5/3} $\to$ 1.6666666666666667
        \item[$\%$] \texttt{5\%3} $\to$ 2, \texttt{-5\%3} $\to$ -2, \texttt{5\%-3} $\to$ 2 
        \item[$**$] \texttt{5**3} $\to$ \texttt{125} 
    \end{itemize}
    \hiv{\href{https://developer.mozilla.org/en-US/docs/Web/JavaScript/Reference/Operators/Operator_Precedence\#table}{Precedencia táblázat}}
\end{frame}

\subsection{Karakterláncok}

\begin{frame}
    Jellemzők
    \begin{itemize}
        \item Unicode, 16 bit karakterenként
        \item Nincs specifikus típus egyetlen karakter tárolására
        \item Jelölés: \texttt{'}-ok vagy \texttt{"}-ek között
        \item Pl. \texttt{'JavaScript'}, \texttt{"JavaScript"}, \texttt{"Guns 'n' Roses"}, \texttt{"Egy\textbackslash nKettő\textbackslash nHárom"}, \texttt{'Guns \textbackslash 'n\textbackslash' Roses'}, \texttt{"Új sor \textbackslash\textbackslash n megadásával kérhető."}
        \item \emph{Template literal:} \texttt{`}-ek között, kifejezések kiértékelése
        \item Pl. \texttt{`5 * 3 = \$\{5*3\}`} $\to$ \texttt{"5 * 3 = 15"}
    \end{itemize}
    \vfill
    Operátor
    \begin{itemize}
        \item[$+$] \texttt{"Java" + 'Script'} $\to$ \texttt{"JavaScript"}
    \end{itemize}
\end{frame}

\subsection{Logikai értékek}

\begin{frame}
    Jellemzők
    \begin{itemize}
        \item Értékek: \texttt{true}, \texttt{false}
        \item Pl. \texttt{5 < 3} $\to$ \texttt{false}
    \end{itemize}
    \vfill
    Logikai operátorok
    \begin{itemize}
        \item[és] \texttt{true \&\& false} $\to$ \texttt{false}
        \item[vagy] \texttt{true || false} $\to$ \texttt{true}
        \item[nem] \texttt{!true} $\to$ \texttt{false}
    \end{itemize}
    \vfill
    \emph{Short circuit evaluation} (pl. alapérték megadására):\\ \qquad \texttt{undefined || "Gizi"} $\to$ \texttt{"Gizi"}, \texttt{null || "Gizi"} $\to$ \texttt{"Gizi"}, \\ \qquad \texttt{"" || "Gizi"} $\to$ \texttt{"Gizi"}, \texttt{"Gizi" || "Mari"} $\to$ \texttt{"Gizi"}
\end{frame}

\begin{frame}
    Relációs operátorok
    \begin{itemize}
        \item \texttt{==}, \texttt{!=}, \texttt{<}, \texttt{<=}, \texttt{>}, \texttt{>=} 
        \item Pl. \texttt{"Bill" != "Gates"} $\to$ \texttt{true}, \texttt{Infinity == Infinity} $\to$ \texttt{true}, \\ \kiemel{de} \texttt{NaN == NaN} $\to$ \texttt{false}
        \item Karakterláncok összehasonlítása: karakterkódok alapján
    \end{itemize}
\end{frame}

\subsection{Egyebek}

\begin{frame}
    Üres értékek: valaminek a hiányát jelzik
    \begin{itemize}
        \item \texttt{undefined}
        \item \texttt{null}
    \end{itemize}
    \vfill
    Egyoperandusú operátorok
    \begin{itemize}
        \item[típus] \texttt{typeof(5)} $\to$ \texttt{"number"}, \texttt{typeof("Gizi")} $\to$ \texttt{"string"}
        \item[$-$] \texttt{-(5)} $\to$ \texttt{-5}
    \end{itemize}
    \vfill
    Háromoperandusú operátor
    \begin{itemize}
        \item[?:] \texttt{1<2?"kisebb":"nagyobb"} $\to$ \texttt{"kisebb"}
    \end{itemize}
\end{frame}

\subsection{Automatikus típuskonverzió (\emph{Type coercion})}

\begin{frame}
    Néhány példa:
    \begin{itemize}
        \item \texttt{5 * null} $\to$ \texttt{0}
        \item \texttt{"5" - 3} $\to$ \texttt{2}
        \item \texttt{"5" + 3} $\to$ \texttt{"53"}
        \item \texttt{"öt" * 3} $\to$ \texttt{NaN}, \texttt{5 * undefined} $\to$ \texttt{NaN}
        \item \texttt{false == 0} $\to$ \texttt{true}, \texttt{true == 1} $\to$ \texttt{true}, \texttt{true == 2} $\to$ \texttt{false}, \item \texttt{"" == false} $\to$ \texttt{true}
        \item Definiált az érték? \texttt{null == undefined} $\to$ \texttt{true}, \texttt{null == 0} $\to$ \texttt{false}
    \end{itemize}
    \vfill
    Típusok egyezését megkövetelő operátorok: \texttt{===}, \texttt{!==}
\end{frame}

\section{Változók, konstansok, vezérlési szerkezetek}

\subsection{Változók és konstansok}

\begin{frame}
    Változók (\emph{variable}, \emph{binding})
    \begin{itemize}
        \item Deklaráció: \texttt{let} (blokk hatáskör), \texttt{var} (függvény hatáskör)
        \item Inkább tekinthető értékre mutató referenciának, mint valódi tárolónak
    \end{itemize}
    \begin{exampleblock}{Példa}
        a $\to$ ReferenceError: a is not defined\\
        let a\\
        a $\to$ undefined\\
        a = 5\\
        a $\to$ 5\\
        let b = 3, c\\
        a * b $\to$ 15\\
    \end{exampleblock}
\end{frame}

\begin{frame}
    Konstansok
    \begin{itemize}
        \item \texttt{const}
    \end{itemize}
    \begin{exampleblock}{Példa}
        const c = 3.14\\
        c = 2 $\to$ TypeError: invalid assignment to const 'c'
    \end{exampleblock}
    \vfill
    Névadási szabályok
    \begin{itemize}
        \item betűket, számokat, \$ és \_ karaktereket tartalmazhat
        \item számjeggyel nem kezdődhet
        \item nem lehet foglalt szó (pl. \texttt{let})
        \item kis- és nagybetűket megkülönbözteti
        \item javasolt stílus: \emph{camel case} (\texttt{hosszuValtozoNeve})
    \end{itemize}
\end{frame}

\begin{frame}
    Változókkal használható (összetett és unáris) operátorok
    \begin{itemize}
        \item \texttt{+=}, \texttt{-=}, \texttt{*=}, \texttt{/=}, \texttt{\%=}, \texttt{\&\&=}, \texttt{||=}, \texttt{**=}, \dots 
        \item \texttt{++}, \texttt{--}
    \end{itemize}
    \vfill
    Környezet (\emph{environment})
    \begin{itemize}
        \item adott pillanatban létező változók és értékeik
        \item gyakorlatilag soha nincs üres környezet
    \end{itemize}
    \vfill
    Megjegyzések
    \begin{itemize}
        \item // egysoros
        \item /* több \\ \quad soros */
    \end{itemize}
\end{frame}

\subsection{Vezérlési szerkezetek}

\begin{frame}
    Szelekció
    \begin{columns}[T]
        \column{0.5\textwidth}
        \begin{itemize}
          \item \texttt{if}(\emph{feltétel}) \emph{utasítás};
          \item \texttt{if}(\emph{feltétel}) \{\\
          \qquad // utasítások \\
          \}
          \item \texttt{if}(\emph{feltétel}) \{ \\
            \qquad // igaz ág utasításai \\
            \} \texttt{else} \{ \\
            \qquad // hamis ág utasításai \\
            \}
        \end{itemize}
        \column{0.5\textwidth}
        \begin{itemize}
          \item Mikor \kiemel{nem} teljesül a \emph{feltétel}?
          \begin{itemize}
            \item false
            \item 0
            \item ""
            \item NaN
            \item null
            \item undefined
          \end{itemize}
        \end{itemize}
      \end{columns}
\end{frame}

\begin{frame}
    Több irányú elágazás
    \begin{columns}[T]
        \column{0.4\textwidth}
        \texttt{switch}(\emph{kifejezés}) \{ \\
        \qquad \texttt{case} \emph{érték1}: \\
        \qquad \qquad // utasítások \\
        \qquad \qquad \texttt{break}; \\
        \qquad \texttt{case} \emph{érték2}: \\
        \qquad \texttt{case} \emph{érték3}: \\
        \qquad \qquad // utasítások \\
        \qquad \qquad \texttt{break}; \\
        \qquad \texttt{default}: \\
        \qquad \qquad // utasítások \\
        \qquad \qquad \texttt{break}; \\
        \}
      \column{0.5\textwidth}
        Az értéknek \emph{és} a típusnak is egyeznie kell! \\
        A \emph{default} ág elhagyható.
    \end{columns}
\end{frame}

\begin{frame}
    Ciklusok
    \vfill
    \texttt{for}(\emph{előkészítés}; \emph{ismétlési\_feltétel}; \emph{frissítés}) \{ \\
    \qquad // Ciklusmag utasításai \\
    \}
    \vfill
    \texttt{while}(\emph{ismétlési\_feltétel}) \{ \\
    \qquad // Ciklusmag utasításai \\
    \}
    \vfill
    \texttt{do} \{ \\
    \qquad // Ciklusmag utasításai \\
    \} \texttt{while} (\emph{ismétlési\_feltétel});
    \vfill
    \texttt{break}, \texttt{continue}
\end{frame}

\begin{frame}[fragile]
    Feladatok
    \begin{exampleblock}{Háromszög rajzolás (\textattachfile{haromszog.js}{megoldás})}
        A böngésző JavaScript konzolján egy sornyi szöveget a \texttt{console.log()} hívással tud megjeleníteni. Használja ezt a következő háromszög megrajzolására:\\
        \begin{verbatim}
*
**
***
****
*****                
\end{verbatim}
    \end{exampleblock}
\end{frame}

\begin{frame}[fragile]
    \begin{exampleblock}{X rajzolás (\textattachfile{x.js}{megoldás})}
        Most rajzoljon 5x5-ös méretű X-et csillagokból:\\
        \begin{verbatim}
*   *
 * *
  *
 * *
*   *            
\end{verbatim}
    \end{exampleblock}
\end{frame}

\begin{frame}[fragile]
    \begin{exampleblock}{Sakktábla (\textattachfile{sakk.js}{megoldás})}
        Rajzoljon meg egy 8x8-as méretű sakktáblát, szintén csillagokból!\\
        \begin{verbatim}
 * * * *
* * * *
 * * * *
* * * *
 * * * *
* * * *
 * * * *
* * * *
\end{verbatim}
    \end{exampleblock}
\end{frame}

\begin{frame}[fragile]
    \begin{exampleblock}{FizzBuzz (\textattachfile{fizzbuzz.js}{megoldás})}
        Vizsgálja meg az egész számokat 1-től 100-ig, majd a vizsgálat eredményét jelenítse meg egymás alatti sorokban! Ha a szám osztható 3-mal, írja ki, hogy \emph{Fizz}, ha 5-tel osztható, akkor azt, hogy \emph{Buzz}, ha pedig 3-mal és 5-tel is osztható, akkor azt, hogy \emph{FizzBuzz}! Ha egyik számmal sem osztható, akkor írja ki a vizsgált számot!\\
        \begin{verbatim}
1
2
Fizz
4
Buzz
Fizz
...
\end{verbatim}
    \end{exampleblock}
\end{frame}

\end{document}