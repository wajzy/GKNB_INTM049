\section{Néhány könyvtári objektum}

\subsection{A \texttt{String} objektum}

\begin{frame}
  Főbb jellemzők:
  \begin{itemize}
    \item Immutable object (mint Java-ban)
    \item Létrehozás literálként: \kiemel{\texttt{'}}-ok vagy \kiemel{\texttt{"}}-ek között
  \end{itemize}
  Nyilvános tulajdonság:
  \begin{description}[m]
    \item[\texttt{length}] \hfill \\ A karakterlánc hossza
  \end{description}
  Metódusok
  \begin{description}[m]
    \item[\texttt{charAt()}, \texttt{[ ]}] \hfill \\ Adott indexű karakter lekérdezése
    \item[\texttt{indexOf(\emph{keresett}[, \emph{tol}])}, \texttt{lastIndexOf(\emph{keresett}[, \emph{tol}])}] \hfill \\ Rész-karakterlánc (\texttt{\emph{keresett}}) első/utolsó előfordulásának keresése \texttt{\emph{tol}} indexű helytől kezdve. \texttt{indexOf}-nál negatív index is támogatott. Ha nincs találat, a visszatérési érték $-1$.
  \end{description}
\end{frame}

\begin{frame}
  \begin{exampleblock}{\textattachfile{string.js}{Adott indexű karakter lekérése, rész-karakterlánc keresése}}
    \lstinputlisting[language=JavaScript,numbers=left,linerange=1-8,firstnumber=1]{string.js}
  \end{exampleblock}
\end{frame}

\begin{frame}
  \begin{description}[m]
    \item[\texttt{slice(\emph{tol}[, \emph{ig}])}] \hfill \\ A [\texttt{\emph{tol}}, \texttt{\emph{ig}}) index intervallumba eső karaktersorozat visszaadása. Negatív indexek támogatottak.
    \begin{exampleblock}{\textattachfile{string.js}{Rész-karakterlánc visszaadása}}
    \lstinputlisting[language=JavaScript,numbers=left,linerange=10-13,firstnumber=10]{string.js}
  \end{exampleblock}
  \end{description}
\end{frame}

\begin{frame}
  \begin{description}[m]
    \item[\texttt{concat(\emph{s1}[, \emph{s2}[, \dots[, \emph{sN}]]])}, \texttt{$+$}, \texttt{$+=$}] \hfill \\ Karakterláncok összefűzése. Az operátorok gyorsabban működnek.
    \item[\texttt{toLowerCase()}] \hfill \\ Kisbetűs alak előállítása.
    \item[\texttt{toUpperCase()}] \hfill \\ Nagybetűs alak előállítása.
    \footnotesize
    \begin{exampleblock}{\textattachfile{string.js}{Összefűzés, kis- és nagybetűs alakra alakítás}}
    \lstinputlisting[language=JavaScript,numbers=left,linerange=15-20,firstnumber=15]{string.js}
  \end{exampleblock}
  \end{description}
\end{frame}

\begin{frame}
  \begin{description}[m]
    \small
    \item[\texttt{trimStart()}, \texttt{trimEnd()}, \texttt{trim()}] \hfill \\ 
    Fehér karakterek eltávolítása egy karakterlánc elejéről, végéről, vagy mindkét végéről.
    \item[\texttt{split([\emph{elvalaszto}[, \emph{max}]])}] \hfill \\ Karakterlánc szétdarabolása, \texttt{\emph{elvalaszto}} jelek mentén (vagy reguláris kifejezéssel) és a darabok visszadása tömbben. \texttt{\emph{max}} korlátozhatja a tömb méretét.
    \item[\texttt{join([\emph{elvalaszto}])}] \hfill \\ A \kiemel{tömb} metódusa, mellyel elemei egyetlen karakterlánccá összefűzhetőek.
    \scriptsize
    \begin{exampleblock}{\textattachfile{string.js}{Fehér karakterek levágása, darabolás és összefűzés}}
    \vspace{-.3cm}
    \lstinputlisting[language=JavaScript,numbers=left,linerange=22-30,firstnumber=22]{string.js}
    \vspace{-.3cm}
  \end{exampleblock}
  \end{description}
\end{frame}

\begin{frame}
  \begin{description}[m]
    \item[\texttt{padStart(\emph{hossz}[, \emph{kitolto}])}, \texttt{padEnd(\emph{hossz}[, \emph{kitolto}])}] \hfill \\ 
    Karakterlánc meghosszabbítása \texttt{\emph{kitolto}} karakterrel balról vagy jobbról \texttt{\emph{hossz}} hosszúságúra.
    \item[\texttt{repeat(\emph{db})}] \hfill \\ Egymás után fűzés \texttt{\emph{db}} alkalommal.
    \footnotesize
    \begin{exampleblock}{\textattachfile{string.js}{Kitöltés, ismétlés}}
    \lstinputlisting[language=JavaScript,numbers=left,linerange=32-35,firstnumber=32]{string.js}
  \end{exampleblock}
  \end{description}
\end{frame}

\subsection{A \texttt{Math} objektum}

\begin{frame}
  Statikus metódusok ($\approx$ a \texttt{Math} objektum csak egy névtér):
  \begin{description}[m]
    \item[\texttt{abs(\emph{n})}] \hfill \\ \emph{n} abszolút értékét adja vissza
    \item[\texttt{floor(\emph{n})}, \texttt{ceil(\emph{n})}] \hfill \\ \emph{n}-nél kisebb/nagyobb egészek közül a legnagyobbat/legkisebbet adja
    \item[\texttt{round(\emph{n})}] \hfill \\ \emph{n}-et a legközelebbi egészre kerekíti
    \item[\texttt{min(\emph{v1}, \emph{v2}, \dots, \emph{vn})}, \texttt{max(\emph{v1}, \emph{v2}, \dots, \emph{vn})}] \hfill \\ a paraméterek közül a legkisebbet/legnagyobbat adja vissza
    \item[\texttt{pow(\emph{alap}, \emph{kitevo})}] \hfill \\ \emph{alap}-ot \emph{kitevo}-re emeli
    \item[\texttt{sqrt(\emph{n})}] \hfill \\ \emph{n} négyzetgyökét adja
  \end{description}
\end{frame}

\begin{frame}
  \begin{description}[m]
    \item[\texttt{random()}] \hfill \\ álvéletlen szám a [0; 1) intervallumból
    \item[\texttt{sin()}, \texttt{cos()}, \texttt{tan()}] \hfill \\ trigonometrikus függvények (paraméterek radiánban!)
    \item[\texttt{asin()}, \texttt{acos()}, \texttt{atan()}] \hfill \\ trigonometrikus függvények inverz függvényei
    \item[\texttt{exp()}, \texttt{log()}] \hfill \\ exponenciális fv., természetes alapú logaritmus
  \end{description}
  Konstansok
  \begin{description}[\texttt{PI}]
    \item[\texttt{PI}] 3.1415\dots
    \item[\texttt{E}] 2.71\dots
  \end{description}
\end{frame}

\begin{frame}
  Cinkelt kocka: a paraméterekkel megadható, hogy
  \begin{itemize}
    \item hány oldala van a kockának, és
    \item ezek dobási valószínűségét súlyokkal lehet befolyásolni
  \end{itemize}
  Függvény definíció és hívás egy lépésben
  \begin{exampleblock}{\textattachfile{cinkeltKocka.js}{Cinkelt kocka}}
    \footnotesize
    \vspace{-.3cm}
    \lstinputlisting[language=JavaScript,numbers=left]{cinkeltKocka.js}
    \vspace{-.3cm}
  \end{exampleblock}
\end{frame}

\begin{frame}
  Átváltás fokról radiánra
  \begin{exampleblock}{\textattachfile{deg2rad.js}{fok $\to$ radián}}
    \lstinputlisting[language=JavaScript,numbers=left]{deg2rad.js}
  \end{exampleblock}
\end{frame}
