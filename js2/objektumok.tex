\section{Objektumokról részletesebben}

\subsection{\texttt{this}}

\begin{frame}
    Objektumok: 
    \begin{itemize}
        \item egységbe zárt (encapsulation) adattagok és metódusok (függvények)
        \item a rejtett \texttt{this} kötés kapcsolja össze az objektum tulajdonságait
        \item minden tulajdonság nyilvános
    \end{itemize}
    \begin{exampleblock}{\textattachfile{teglalap.js}{Objektum literál}}
        \small
        \lstinputlisting[language=JavaScript,numbers=left,linerange=1-7]{teglalap.js}
    \end{exampleblock}
\end{frame}

\begin{frame}
    \begin{exampleblock}{\textattachfile{teglalap.js}{Metódusok}}
        \scriptsize
        \lstinputlisting[language=JavaScript,numbers=left,firstnumber=8,linerange=8-19]{teglalap.js}
    \end{exampleblock}
\end{frame}

\begin{frame}
    \begin{exampleblock}{\textattachfile{teglalap.js}{A \texttt{call} metódus}}
        \scriptsize
        \lstinputlisting[language=JavaScript,numbers=left,firstnumber=21,linerange=21-28]{teglalap.js}
    \end{exampleblock}
\end{frame}

\begin{frame}
    \begin{exampleblock}{\textattachfile{teglalap.js}{Nyíl függvények és \texttt{function} közti különbségek}}
        \small
        \lstinputlisting[language=JavaScript,numbers=left,firstnumber=30,linerange=30-41]{teglalap.js}
    \end{exampleblock}
\end{frame}

\subsection{Prototípusok}

\begin{frame}
    JavaScriptben nincsenek osztályok: az objektumok más objektumok (prototípusok) alapján jönnek létre, azt kiegészítve.\\
    A prototípusnak is lehet prototípusa: származtatási hierarchia, fa struktúra, csúcsán: \texttt{Object.prototype}; ez a közös ős.
    \begin{exampleblock}{\textattachfile{prototipus.js}{Prototípusok felderítése}}
        \scriptsize
        \lstinputlisting[language=JavaScript,numbers=left,firstnumber=2, linerange=2-5]{prototipus.js}
    \end{exampleblock}
\end{frame}

\begin{frame}
    \begin{exampleblock}{\textattachfile{prototipus.js}{Objektum létrehozása alapértelmezett prototípussal}}
        \lstinputlisting[language=JavaScript,numbers=left,firstnumber=7,linerange=7-16]{prototipus.js}
    \end{exampleblock}
\end{frame}

\begin{frame}
    \scriptsize
    \begin{exampleblock}{\textattachfile{prototipus.js}{Objektum létrehozása adott prototípussal}}
        \scriptsize
        \vspace{-.3cm}
        \lstinputlisting[language=JavaScript,numbers=left,firstnumber=18,linerange=18-35]{prototipus.js}
        \vspace{-.3cm}
    \end{exampleblock}
\end{frame}

\subsection{Konstruktorok}

\begin{frame}
    \footnotesize
    Prototípusnak szánt objektumba jellemzően csak olyan tulajdonságot tesznek, melynek értékét minden ebből 
    származó objektumnak tartalmaznia kell. \\
    Konstruktor: új objektum létrehozása adott prototípusból,
    az egyedre jellemző értékek hozzáadásával.
    \begin{exampleblock}{\textattachfile{konstruktor1.js}{Objektum létrehozása konstruktorral}}
        \scriptsize
        \vspace{-.2cm}
        \lstinputlisting[language=JavaScript,numbers=left,linerange=1-14]{konstruktor1.js}
        \vspace{-.2cm}
    \end{exampleblock}
\end{frame}

\begin{frame}
    \begin{exampleblock}{\textattachfile{konstruktor1.js}{Objektum létrehozása konstruktorral}}
        \small
        \lstinputlisting[language=JavaScript,numbers=left,firstnumber=16,linerange=16-21]{konstruktor1.js}
    \end{exampleblock}
\end{frame}

\begin{frame}
    \emph{Majdnem} ugyanez történik, ha a \texttt{new} kulcsszót írjuk egy függvény elé, azaz konstruktorként fog viselkedni.
    \begin{exampleblock}{\textattachfile{konstruktor2.js}{Objektum létrehozása \texttt{new} kulcsszóval}}
        \footnotesize
        \lstinputlisting[language=JavaScript,numbers=left,linerange=1-11]{konstruktor2.js}
    \end{exampleblock}
\end{frame}

\begin{frame}
    \begin{exampleblock}{\textattachfile{konstruktor2.js}{Objektum létrehozása \texttt{new} kulcsszóval}}
        \footnotesize
        \lstinputlisting[language=JavaScript,numbers=left,firstnumber=12,linerange=12-19]{konstruktor2.js}
    \end{exampleblock}
    Minden függvénynek van \texttt{prototype} tulajdonsága, a konstruktornak is. Ez egy lecserélhető üres objektum, 
    de a meglévőhöz is hozzáadhatók új tulajdonságok, mint a példában. 
\end{frame}

\subsection{\texttt{class} és \texttt{constructor}}

\begin{frame}
    2015-től lehet használni a \texttt{class} és \texttt{constructor} kulcsszavakat. A háttérben ettől még 
    ugyanaz történik (syntactic sugar), továbbra sem léteznek valódi osztályok a nyelvben. Konstans adatokat nem lehet 
    az objektumhoz adni.
    \begin{exampleblock}{\textattachfile{class.js}{\texttt{class}, \texttt{constructor}}}
        \footnotesize
        \vspace{-.2cm}
        \lstinputlisting[language=JavaScript,numbers=left,linerange=1-11]{class.js}
        \vspace{-.2cm}
    \end{exampleblock}
\end{frame}

\begin{frame}
    \begin{exampleblock}{\textattachfile{class.js}{\texttt{class}, \texttt{constructor}}}
        \footnotesize
        \lstinputlisting[language=JavaScript,numbers=left,firstnumber=13,linerange=13-22]{class.js}
    \end{exampleblock}
\end{frame}

\begin{frame}
    A \texttt{class} kifejezésben is szerepelhet, nem csak utasításban.
    \begin{exampleblock}{\textattachfile{class.js}{\texttt{class} kifejezés}}
        \small
        \lstinputlisting[language=JavaScript,numbers=left,firstnumber=24,linerange=24-33]{class.js}
    \end{exampleblock}
\end{frame}

\begin{frame}
    Az \texttt{Object}-től örökölt metódusok egyedileg felüldefiniálhatók.
    \begin{exampleblock}{\textattachfile{class.js}{Felüldefiniálás}}
        \small
        \lstinputlisting[language=JavaScript,numbers=left,firstnumber=35,linerange=35-42]{class.js}
    \end{exampleblock}
\end{frame}

\subsection{Objektumok használata asszociatív tömbként}

\begin{frame}
    \footnotesize
    Objektumokat használni asszociatív tömbként nem biztonságos:
    \begin{enumerate}
        \item öröklött tulajdonságok is megjelennek kulcsként
        \item implicit típuskonverzió miatt kulcsnak látszódhat egy érték, ami nem az
    \end{enumerate}
    \begin{exampleblock}{\textattachfile{asszociativ.js}{Asszociatív tömb objektummal}}
        \vspace{-.2cm}
        \scriptsize
        \lstinputlisting[language=JavaScript,numbers=left,linerange=1-13]{asszociativ.js}
        \vspace{-.2cm}
    \end{exampleblock}
\end{frame}

\begin{frame}
    Az első problémát megoldja, ha nincs prototípusa az objektumnak.
    \begin{exampleblock}{\textattachfile{asszociativ.js}{Asszociatív tömb objektummal}}
        \scriptsize
        \lstinputlisting[language=JavaScript,numbers=left,firstnumber=15,linerange=15-21]{asszociativ.js}
    \end{exampleblock}
\end{frame}

\begin{frame}
    A másodikon viszont nem segít, sőt mellékhatása is lehet (pl. nyomkövetésnél)
    Megoldás: \texttt{Map} (és \texttt{Set}), ld. később
    \begin{exampleblock}{\textattachfile{asszociativ.js}{Asszociatív tömb objektummal}}
        \lstinputlisting[language=JavaScript,numbers=left,firstnumber=23,linerange=23-27]{asszociativ.js}
    \end{exampleblock}
\end{frame}

\subsection{Szimbólumok}

\begin{frame}
    Az objektumok kulcsaiként eddig mindig \texttt{String}-ek álltak. Ez azonban akkor is lehetővé 
    teszi pl. egy metódus felüldefiniálását vagy lecserélését, ha nem szándékoztuk.
    \begin{exampleblock}{\textattachfile{szimbolumok.js}{Tulajdonságok karakterlánccal megadva}}
        \small
        \vspace{-.2cm}
        \lstinputlisting[language=JavaScript,numbers=left,linerange=1-11]{szimbolumok.js}
        \vspace{-.2cm}
    \end{exampleblock}
\end{frame}

\begin{frame}
    Ezzel szemben a létrehozott szimbólumok mindig egyediek.
    \begin{exampleblock}{\textattachfile{szimbolumok.js}{Tulajdonságok szimbólummal megadva}}
        \scriptsize
        \lstinputlisting[language=JavaScript,numbers=left,firstnumber=13,linerange=13-26]{szimbolumok.js}
    \end{exampleblock}
\end{frame}

\subsection{Iterátorok}

\begin{frame}
    \footnotesize
    A \texttt{for/of} ciklusokkal azok a gyűjtemények járhatók be, melyek megvalósítják az \texttt{iterator} interfészt,
    azaz rendelkeznek egy \texttt{Symbol.iterator} nevű metódussal, ami visszadja a tényleges bejárást biztosító
    iterátor objektumot.
    
    Ennek \texttt{next()} metódusával lehet elkérni a \texttt{value} és \texttt{done} kulcsokkal rendelkező objektumokat. 
    Az első adattag a tényleges, soron következő értéket adja meg, a másodikból kiderül, hogy elérhető-e még egyáltalán további adat.

    Legismertebb megvalósítások a nyelvben: \texttt{Array}, \texttt{String}.
    \begin{exampleblock}{\textattachfile{iterator.js}{Beépített objektumok iterátorral}}
        \scriptsize
        \vspace{-.2cm}
        \lstinputlisting[language=JavaScript,numbers=left,linerange=1-10]{iterator.js}
        \vspace{-.2cm}
    \end{exampleblock}
\end{frame}

\begin{frame}
    Készítsünk a \hiv{\href{https://docs.python.org/3/library/functions.html\#func-range}{Python}} és 
    a \hiv{\href{https://www.php.net/manual/en/function.range.php}{PHP}} mintájára olyan \texttt{Intervallum}
    objektumot, ami a [\texttt{tol}, \texttt{ig}) intervallumból ad vissza egymástól \texttt{lepes}-nyire lévő értékeket!
    \begin{exampleblock}{\textattachfile{iterator.js}{Saját objektum iterátorral}}
        \small
        \lstinputlisting[language=JavaScript,numbers=left,firstnumber=12,linerange=12-21]{iterator.js}
    \end{exampleblock}
\end{frame}

\begin{frame}
    \footnotesize
    \begin{exampleblock}{\textattachfile{iterator.js}{Saját objektum iterátorral}}
        \scriptsize
        \vspace{-.2cm}
        \lstinputlisting[language=JavaScript,numbers=left,firstnumber=23,linerange=23-38]{iterator.js}
        \vspace{-.2cm}
    \end{exampleblock}
\end{frame}

\begin{frame}
    \begin{exampleblock}{\textattachfile{iterator.js}{Saját objektum iterátorral}}
        \lstinputlisting[language=JavaScript,numbers=left,firstnumber=40,linerange=40-45]{iterator.js}
    \end{exampleblock}
\end{frame}

\subsection{getter, setter, static}

\begin{frame}
    JS-ben valamennyi adattag nyilvánosan elérhető. Néha viszont nem szeretnénk (redundáns) adattagokat létrehozni, abban folyamatosan
    adatokat tárolni és azt frissíteni, amikor az objektum változik. Ilyenkor létrehozhatunk olyan metódusokat, melyek adattagnak
    tűnnek a külvilág számára, és a számított adatokat szolgáltatják, amikor szükség van rájuk.
    \vfill
    Hasonlóképpen létrehozhatunk adattagnak tűnő metódusokat tárolt értékek beállítására, input ellenőrzésére is.
    \vfill
    Időnként egy adat nem példányhoz kötődik, hanem az ,,osztályhoz'' (ld. \texttt{Math.PI}). Ezeket célszerű csak egyszer tárolni
    $\to$ statikus adattag. Hasonlóan, ha egy metódus nem dolgozik a példány adataival, megjelölhetjük statikusként (pl. \texttt{Math.sin()}).
    Hívásakor a prototípus/osztály nevével minősítjük. Jellemzően objektumok létrehozására, másolására használják őket.
\end{frame}

\begin{frame}
    \begin{exampleblock}{\textattachfile{gallon.js}{getter, setter, static}}
        \scriptsize
        \lstinputlisting[language=JavaScript,numbers=left,linerange=1-15]{gallon.js}
    \end{exampleblock}
\end{frame}

\begin{frame}
    \begin{exampleblock}{\textattachfile{gallon.js}{getter, setter, static}}
        \footnotesize
        \lstinputlisting[language=JavaScript,numbers=left,firstnumber=17,linerange=17-24]{gallon.js}
    \end{exampleblock}
\end{frame}

\subsection{Származtatás}

\begin{frame}
    Származtatás megvalósításához, az ős megnevezésére megjelent az \texttt{extends} kulcsszó. Az ősre \texttt{super} segítségével lehet
    hivatkozni.
    \begin{exampleblock}{\textattachfile{szarmaztatas.js}{Származtatás}}
        \scriptsize
        \vspace{-.2cm}
        \lstinputlisting[language=JavaScript,numbers=left,linerange=1-14]{szarmaztatas.js}
        \vspace{-.2cm}
    \end{exampleblock}
\end{frame}

\begin{frame}
    \begin{exampleblock}{\textattachfile{szarmaztatas.js}{Származtatás}}
        \scriptsize
        \vspace{-.2cm}
        \lstinputlisting[language=JavaScript,numbers=left,firstnumber=16,linerange=16-31]{szarmaztatas.js}
        \vspace{-.2cm}
    \end{exampleblock}
\end{frame}

\begin{frame}
    \begin{exampleblock}{\textattachfile{szarmaztatas.js}{Származtatás}}
        \small
        \lstinputlisting[language=JavaScript,numbers=left,firstnumber=33,linerange=33-39]{szarmaztatas.js}
    \end{exampleblock}
    Az \texttt{instanceof} operátorral ellenőrizhető egy objektum (akár közvetett) prototípusa.
\end{frame}

\subsection{Feladatok}

\begin{frame}
    \begin{exampleblock}{Lekerekített téglalap (\textattachfile{lekerekitett.js}{lekerekitett.js})}
        Készítsen \texttt{Teglalap} ,,osztályt'', melynek konstruktora megkapja paraméterként a síkidom szélességét, magasságát,
        és egy logikai értéket, melyből kiderül, hogy rajzolásnál csak a körvonalat kell megrajzolni, vagy a téglalap belsejét is
        ki kell tölteni * karakterekkel! A \texttt{rajz()} metódus adja vissza a rajzot egy \texttt{String} formájában!

        Származtasson ebből egy \texttt{Lekerekitett} nevű osztályt, amelynek konstruktora egy további paramétert fogad, a sarkok
        lekerekítési sugarát!
    \end{exampleblock}
\end{frame}