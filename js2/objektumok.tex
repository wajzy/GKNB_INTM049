\section{Objektumokról részletesebben}

\subsection{\texttt{this}}

\begin{frame}
    Objektumok: 
    \begin{itemize}
        \item egységbe zárt (encapsulation) adattagok és metódusok (függvények)
        \item a rejtett \texttt{this} kötés kapcsolja össze az objektum tulajdonságait
        \item minden tulajdonság nyilvános
    \end{itemize}
    \begin{exampleblock}{\textattachfile{teglalap.js}{Objektum literál}}
        \small
        \lstinputlisting[language=JavaScript,numbers=left,linerange=1-7]{teglalap.js}
    \end{exampleblock}
\end{frame}

\begin{frame}
    \begin{exampleblock}{\textattachfile{teglalap.js}{Metódusok}}
        \scriptsize
        \lstinputlisting[language=JavaScript,numbers=left,firstnumber=8,linerange=8-19]{teglalap.js}
    \end{exampleblock}
\end{frame}

\begin{frame}
    \begin{exampleblock}{\textattachfile{teglalap.js}{A \texttt{call} metódus}}
        \small
        \lstinputlisting[language=JavaScript,numbers=left,firstnumber=21,linerange=21-27]{teglalap.js}
    \end{exampleblock}
\end{frame}

\begin{frame}
    \begin{exampleblock}{\textattachfile{teglalap.js}{Nyíl függvények és \texttt{function} közti különbségek}}
        \small
        \lstinputlisting[language=JavaScript,numbers=left,firstnumber=29,linerange=29-40]{teglalap.js}
    \end{exampleblock}
\end{frame}

\begin{frame}
    JavaScriptben nincsenek osztályok: az objektumok más objektumok (prototípusok) alapján jönnek létre, azt kiegészítve.\\
    A prototípusnak is lehet prototípusa: származtatási hierarchia, fa struktúra, csúcsán: \texttt{Object.prototype}; ez a közös ős.
    \begin{exampleblock}{\textattachfile{prototipus.js}{Prototípusok felderítése}}
        \scriptsize
        \lstinputlisting[language=JavaScript,numbers=left,firstnumber=2, linerange=2-5]{prototipus.js}
    \end{exampleblock}
\end{frame}

\begin{frame}
    \begin{exampleblock}{\textattachfile{prototipus.js}{Objektum létrehozása adott prototípussal}}
        \lstinputlisting[language=JavaScript,numbers=left,firstnumber=7,linerange=7-16]{prototipus.js}
    \end{exampleblock}
\end{frame}

\begin{frame}
    \scriptsize
    \begin{exampleblock}{\textattachfile{prototipus.js}{Objektum létrehozása adott prototípussal}}
        \scriptsize
        \vspace{-.3cm}
        \lstinputlisting[language=JavaScript,numbers=left,firstnumber=18,linerange=18-35]{prototipus.js}
        \vspace{-.3cm}
    \end{exampleblock}
\end{frame}

\begin{frame}
    \footnotesize
    Prototípusnak szánt objektumba jellemzően csak olyan tulajdonságot tesznek, amit minden ebből 
    származó objektumnak tartalmaznia kell. \\
    Konstruktor: új objektum létrehozása adott prototípusból,
    az egyedre jellemző értékek hozzáadásával.
    \begin{exampleblock}{\textattachfile{konstruktor1.js}{Objektum létrehozása konstruktorral}}
        \scriptsize
        \vspace{-.2cm}
        \lstinputlisting[language=JavaScript,numbers=left,linerange=1-14]{konstruktor1.js}
        \vspace{-.2cm}
    \end{exampleblock}
\end{frame}

\begin{frame}
    \begin{exampleblock}{\textattachfile{konstruktor1.js}{Objektum létrehozása konstruktorral}}
        \small
        \lstinputlisting[language=JavaScript,numbers=left,firstnumber=16,linerange=16-21]{konstruktor1.js}
    \end{exampleblock}
\end{frame}

\begin{frame}
    \emph{Majdnem} ugyanez történik, ha a \texttt{new} kulcsszót írjuk egy függvény elé, azaz konstruktorként fog viselkedni.
    \begin{exampleblock}{\textattachfile{konstruktor2.js}{Objektum létrehozása \texttt{new} kulcsszóval}}
        \footnotesize
        \lstinputlisting[language=JavaScript,numbers=left,linerange=1-11]{konstruktor2.js}
    \end{exampleblock}
\end{frame}

\begin{frame}
    \begin{exampleblock}{\textattachfile{konstruktor2.js}{Objektum létrehozása \texttt{new} kulcsszóval}}
        \footnotesize
        \lstinputlisting[language=JavaScript,numbers=left,firstnumber=12,linerange=12-19]{konstruktor2.js}
    \end{exampleblock}
    Minden függvénynek van \texttt{prototype} tulajdonsága, a konstruktornak is. Ez egy lecserélhető üres objektum, 
    de a meglévőhöz is hozzáadhatók új tulajdonságok, mint a példában. 
\end{frame}

\begin{frame}
    2015-től lehet használni a \texttt{class} és \texttt{constructor} kulcsszavakat. A háttérben ettől még 
    ugyanaz történik (syntactic sugar), továbbra sem léteznek valódi osztályok a nyelvben. Konstans adatokat nem lehet 
    az objektumhoz adni.
    \begin{exampleblock}{\textattachfile{class.js}{\texttt{class}, \texttt{constructor}}}
        \footnotesize
        \vspace{-.2cm}
        \lstinputlisting[language=JavaScript,numbers=left,linerange=1-11]{class.js}
        \vspace{-.2cm}
    \end{exampleblock}
\end{frame}

\begin{frame}
    \begin{exampleblock}{\textattachfile{class.js}{\texttt{class}, \texttt{constructor}}}
        \footnotesize
        \lstinputlisting[language=JavaScript,numbers=left,firstnumber=13,linerange=13-22]{class.js}
    \end{exampleblock}
\end{frame}

\begin{frame}
    A \texttt{class} kifejezésben is szerepelhet, nem csak utasításban.
    \begin{exampleblock}{\textattachfile{class.js}{\texttt{class}, \texttt{constructor}}}
        \small
        \lstinputlisting[language=JavaScript,numbers=left,firstnumber=24,linerange=24-33]{class.js}
    \end{exampleblock}
\end{frame}

\begin{frame}
    Az \texttt{Object}-től örökölt metódusok egyedileg felüldefiniálhatók.
    \begin{exampleblock}{\textattachfile{class.js}{\texttt{class}, \texttt{constructor}}}
        \small
        \lstinputlisting[language=JavaScript,numbers=left,firstnumber=35,linerange=35-41]{class.js}
    \end{exampleblock}
\end{frame}