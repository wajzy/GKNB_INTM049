\section{Változók, konstansok, vezérlési szerkezetek}

\subsection{Változók és konstansok}

\begin{frame}
    Változók (\emph{variable}, \emph{binding})
    \begin{itemize}
        \item Deklaráció: \texttt{let} (blokk hatáskör), \texttt{var} (függvény hatáskör)
        \item Inkább tekinthető értékre mutató referenciának, mint valódi tárolónak
    \end{itemize}
    \begin{exampleblock}{Példa}
        a \kiemelZ{$\to$ ReferenceError: a is not defined}\\
        let a\\
        a \kiemelZ{$\to$ undefined}\\
        a = 5\\
        a \kiemelZ{$\to$ 5}\\
        let b = 3, c\\
        a * b \kiemelZ{$\to$ 15}\\
    \end{exampleblock}
\end{frame}

\begin{frame}
    Konstansok
    \begin{itemize}
        \item \texttt{const}
    \end{itemize}
    \begin{exampleblock}{Példa}
        const c = 3.14\\
        c = 2 \kiemelZ{$\to$ TypeError: invalid assignment to const 'c'}
    \end{exampleblock}
    \vfill
    Névadási szabályok
    \begin{itemize}
        \item betűket, számokat, \$ és \_ karaktereket tartalmazhat
        \item számjeggyel nem kezdődhet
        \item nem lehet foglalt szó (pl. \texttt{let})
        \item kis- és nagybetűket megkülönbözteti
        \item javasolt stílus: \emph{camel case} (\texttt{hosszuValtozoNeve})
    \end{itemize}
\end{frame}

\begin{frame}
    Változókkal használható (összetett és unáris) operátorok
    \begin{itemize}
        \item \texttt{+=}, \texttt{-=}, \texttt{*=}, \texttt{/=}, \texttt{\%=}, \texttt{\&\&=}, \texttt{||=}, \texttt{**=}, \dots 
        \item \texttt{++}, \texttt{-{-}}
    \end{itemize}
    \vfill
    Környezet (\emph{environment})
    \begin{itemize}
        \item adott pillanatban létező változók és értékeik
        \item gyakorlatilag soha nincs üres környezet
    \end{itemize}
    \vfill
    Megjegyzések
    \begin{itemize}
        \item // egysoros
        \item /* több \\ \quad soros */
    \end{itemize}
\end{frame}

\subsection{Vezérlési szerkezetek}

\begin{frame}
    Szelekció
    \begin{columns}[T]
        \column{0.5\textwidth}
        \begin{itemize}
          \item \texttt{if}(\emph{feltétel}) \emph{utasítás};
          \item \texttt{if}(\emph{feltétel}) \{\\
          \qquad // utasítások \\
          \}
          \item \texttt{if}(\emph{feltétel}) \{ \\
            \qquad // igaz ág utasításai \\
            \} \texttt{else} \{ \\
            \qquad // hamis ág utasításai \\
            \}
        \end{itemize}
        \column{0.5\textwidth}
        \begin{itemize}
          \item Mikor \kiemel{nem} teljesül a \emph{feltétel}?
          \begin{itemize}
            \item false
            \item 0
            \item ""
            \item NaN
            \item null
            \item undefined
          \end{itemize}
        \end{itemize}
      \end{columns}
\end{frame}

\begin{frame}
    Több irányú elágazás
    \begin{columns}[T]
        \column{0.4\textwidth}
        \texttt{switch}(\emph{kifejezés}) \{ \\
        \qquad \texttt{case} \emph{érték1}: \\
        \qquad \qquad // utasítások \\
        \qquad \qquad \texttt{break}; \\
        \qquad \texttt{case} \emph{érték2}: \\
        \qquad \texttt{case} \emph{érték3}: \\
        \qquad \qquad // utasítások \\
        \qquad \qquad \texttt{break}; \\
        \qquad \texttt{default}: \\
        \qquad \qquad // utasítások \\
        \qquad \qquad \texttt{break}; \\
        \}
      \column{0.5\textwidth}
        Az értéknek \emph{és} a típusnak is egyeznie kell! \\
        A \emph{default} ág elhagyható.
    \end{columns}
\end{frame}

\begin{frame}
    Ciklusok
    \vfill
    \texttt{for}(\emph{előkészítés}; \emph{ismétlési\_feltétel}; \emph{frissítés}) \{ \\
    \qquad // Ciklusmag utasításai \\
    \}
    \vfill
    \texttt{while}(\emph{ismétlési\_feltétel}) \{ \\
    \qquad // Ciklusmag utasításai \\
    \}
    \vfill
    \texttt{do} \{ \\
    \qquad // Ciklusmag utasításai \\
    \} \texttt{while} (\emph{ismétlési\_feltétel});
    \vfill
    \texttt{break}, \texttt{continue}
\end{frame}

\subsection{Feladatok}

\begin{frame}[fragile]
    \begin{exampleblock}{Háromszög rajzolás (\textattachfile{haromszog.js}{megoldás})}
        A böngésző JavaScript konzolján egy sornyi szöveget a \texttt{console.log()} hívással tud megjeleníteni. Használja ezt a következő háromszög megrajzolására:\\
        \begin{verbatim}
*
**
***
****
*****                
\end{verbatim}
    \end{exampleblock}
\end{frame}

\begin{frame}[fragile]
    \begin{exampleblock}{X rajzolás (\textattachfile{x.js}{megoldás})}
        Most rajzoljon 5x5-ös méretű X-et csillagokból:\\
        \begin{verbatim}
*   *
 * *
  *
 * *
*   *            
\end{verbatim}
    \end{exampleblock}
\end{frame}

\begin{frame}[fragile]
    \begin{exampleblock}{Sakktábla (\textattachfile{sakk.js}{megoldás})}
        Rajzoljon meg egy 8x8-as méretű sakktáblát, szintén csillagokból!\\
        \begin{verbatim}
 * * * *
* * * *
 * * * *
* * * *
 * * * *
* * * *
 * * * *
* * * *
\end{verbatim}
    \end{exampleblock}
\end{frame}

\begin{frame}[fragile]
    \begin{exampleblock}{FizzBuzz (\textattachfile{fizzbuzz.js}{megoldás})}
        Vizsgálja meg az egész számokat 1-től 100-ig, majd a vizsgálat eredményét jelenítse meg egymás alatti sorokban! Ha a szám osztható 3-mal, írja ki, hogy \emph{Fizz}, ha 5-tel osztható, akkor azt, hogy \emph{Buzz}, ha pedig 3-mal és 5-tel is osztható, akkor azt, hogy \emph{FizzBuzz}! Ha egyik számmal sem osztható, akkor írja ki a vizsgált számot!\\
        \begin{verbatim}
1
2
Fizz
4
Buzz
Fizz
...
\end{verbatim}
    \end{exampleblock}
\end{frame}
