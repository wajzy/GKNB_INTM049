\subsection{Előnyök}

%2
\begin{frame}
  CSS: Cascading Style Sheets
  \begin{itemize}
    \item $\approx$ lépcsőzetes/sorba kapcsolt stíluslapok
    \item \emph{formázás, megjelenés} leírásának elválasztása a \emph{tartalomtól} (HTML), előnyei:
    \begin{itemize}
      \item külön fájlban tárolható, ami több weboldalhoz is használható, így csökken az összesített kódméret, 
      \item egységessé válik ezen oldalak megjelenése,
      \item egymástól függetlenül, egyidejűleg lehet szerkeszteni a formát és a tartalmat,
      \item gyorsabban módosítható a megjelenés, mert csak egy helyen kell változtatni,
      \item hatékonyabbá válik a gyorstárazás,
    \end{itemize}
    \item különféle médiára eltérő formázás lehetséges (pl. képernyő, nyomtatás)
    \item \hiv{\href{http://www.csszengarden.com/}{a CSS ereje}}
    \item \hiv{\href{https://www.w3.org/Style/CSS/learning.en.html}{hivatalos W3C oldal}}
  \end{itemize}
\end{frame}

\subsection{Formázás HTML elemekkel és CSS-sel}

%3
\begin{frame}
  \small
  \begin{alertblock}{Elavult módszer (\textattachfile{htmlFormazas.html}{htmlFormazas.html})}
    \lstinputlisting[style=HTML,numbers=left]{htmlFormazas.html}
  \end{alertblock}
\end{frame}

%4
\begin{frame}
  \scriptsize
  \begin{exampleblock}{Formázás CSS-sel (\textattachfile{cssFormazas.html}{cssFormazas.html})}
    \lstinputlisting[style=HTML,linerange={3-10},numbers=left,firstnumber=3]{cssFormazas.html}
  \end{exampleblock}
  \begin{exampleblock}{Formázás CSS-sel (\textattachfile{cssFormazas.css}{cssFormazas.css})}
    \lstinputlisting[style=HTML,numbers=left]{cssFormazas.css}
  \end{exampleblock}
\end{frame}

\subsection{Szintaktika}

%5
\begin{frame}[fragile]
  \begin{columns}[c]
    \column{0.5\textwidth}
      \includegraphics[scale=0.75]{szintakszis.pdf}\\
      \begin{block}{Deklaráció sablonja}
      \vspace{-0.5cm}
\begin{verbatim}
szelektor {
  tulajdonság1: érték(ek);
  tulajdonság2: érték(ek);
  ...
  tulajdonságN: érték(ek);
}
\end{verbatim}
      \vspace{-0.4cm}
      \end{block}
    \column{0.5\textwidth}
      \begin{description}[m]
        \item[Szelektor] \hfill \\ Mit akarunk formázni?
        \item[Tulajdonság] \hfill \\ Milyen tulajdonságán változtassunk?
        \item[Érték] \hfill \\ Milyen legyen az új állapot?
      \end{description}
  \end{columns}
  
\end{frame}

%6
\begin{frame}
  Megjegyzések a CSS-ben:
  \begin{itemize}
    \item \texttt{/* megjegyzes */}
    \item végleges kódból célszerű elhagyni
    \item Lehet több soros is
  \end{itemize} 
  \hiv{\href{https://jigsaw.w3.org/css-validator/}{CSS ellenőrző}}
\end{frame}

\subsection{Egyszerű szelektorok}

%7
\begin{frame}
  \begin{description}[m]
    \item[HTML elem neve] \hfill \\ \texttt{p \{ font-style: italic; \}}
    \item[Egyedi azonosító (\texttt{id} attribútum) alapján] \hfill \\ 
      \texttt{\#lablec \{ font-size: 10pt; \}}\\
      Az \texttt{id} nem kezdődhet számjegy karakterrel!
    \item[Univerzális szelektor, mindenre illeszkedik] \hfill \\ \texttt{* \{ font-size: smaller; \}}
  \end{description}
\end{frame}

%8
\begin{frame}
  \begin{description}[m]
    \item[Osztály (\texttt{class} attribútum alapján)] \hfill \\ 
      \texttt{*.kisbetus \{ font-size: small; \} /* bármilyen HTML elemhez */} \\
      \texttt{.kisbetus \{ font-size: small; \} /* bármilyen HTML elemhez, rövid alak */}\\
      \texttt{p.voros \{ color: red; \} /* csak adott (pl. <p>) HTML elemhez */}\\
      A \texttt{class} értéke nem kezdődhet számjeggyel, de lehet egyszerre több, szóközzel elválasztott értéke: \\
      \texttt{<p class="kisbetus voros">Apróbetűs piros bekezdés</p>}
    \item[Elemek csoportosítása] \hfill \\ \texttt{h1, h2, h3 \{ font-family: Arial; \}}
  \end{description}
\end{frame}

%9
\begin{frame}
  \begin{exampleblock}{\textattachfile{egyszeruSzelektor.html}{egyszeruSzelektor.html}}
    \scriptsize
    \lstinputlisting[style=HTML,linerange={3-13},numbers=left,firstnumber=3]{egyszeruSzelektor.html}
  \end{exampleblock}
\end{frame}

%10
\begin{frame}
  \begin{exampleblock}{\textattachfile{egyszeruSzelektor.html}{egyszeruSzelektor.html}}
    \scriptsize
    \lstinputlisting[style=HTML,linerange={14-15},numbers=left,firstnumber=14]{egyszeruSzelektor.html}
  \end{exampleblock}
\end{frame}

%11
\begin{frame}
  \begin{exampleblock}{\textattachfile{egyszeruSzelektor.css}{egyszeruSzelektor.css}}
    \scriptsize
    \lstinputlisting[style=HTML,numbers=left]{egyszeruSzelektor.css}
  \end{exampleblock}
  \begin{center}
    \includegraphics[width=.9\textwidth]{egyszeruSzelektor.png}
  \end{center}
\end{frame}

\subsection{Stílusok forrása}

%12
\begin{frame}
  Háromféle helyen lehet stílusokat megadni:
  \begin{enumerate}
    \item Külső fájlban (\texttt{css} kiterjesztés, \texttt{<link>} elem)
    \item A \texttt{<head>} elembe ágyazott \texttt{<style>} elemben. Csak akkor ajánlott, ha egyetlen HTML fájlt kívánunk formázni ezekkel a stílusokkal.
    \item Soron belül: a HTML elemek \texttt{style} attribútumának értékeként. Ismét \kiemel{keveredik a tartalom a stílussal}, ezért általában \kiemel{nem ajánlott} a használata!
  \end{enumerate}
\end{frame}

%13
\begin{frame}
  \begin{exampleblock}{\textattachfile{egyszeruSzelektor2.html}{egyszeruSzelektor2.html}}
    \footnotesize
    \lstinputlisting[style=HTML,linerange={3-12},numbers=left,firstnumber=3]{egyszeruSzelektor2.html}
    \lstinputlisting[style=HTML,linerange={16-16},numbers=left,firstnumber=16]{egyszeruSzelektor2.html}
  \end{exampleblock}
\end{frame}

\subsection{Ütközések feloldása}

%14
\begin{frame}
  Ha több előírás is vonatkozik ugyanannak az objektumnak a formázására, elsőként a forrás prioritása dönt (csökkenő sorrendben):
  \begin{enumerate}
    \item soron belüli formázások
    \item külső és belső (\texttt{<link>}, \texttt{<style>} elemek) formázások
    \item böngésző alapértelmezése
  \end{enumerate}
  Azonos prioritás (pl. két külső stíluslap) esetén a később betöltött szabály felülírja a korábbit.
\end{frame}

%15
\begin{frame}
  \begin{exampleblock}{\textattachfile{utkozes1.html}{utkozes1.html}}
    \scriptsize
    \lstinputlisting[style=HTML,linerange={6-13},numbers=left,firstnumber=6]{utkozes1.html}
  \end{exampleblock}
  \begin{columns}[T]
    \column{0.6\textwidth}
      \begin{exampleblock}{\textattachfile{utkozes1.css}{utkozes1.css}}
        \scriptsize
        \lstinputlisting[style=HTML,numbers=left,firstnumber=1]{utkozes1.css}
      \end{exampleblock}
    \column{0.3\textwidth}
      \includegraphics[width=.66\textwidth]{utkozes1.png}
  \end{columns} 
\end{frame}

%16
\begin{frame}
  \begin{exampleblock}{\textattachfile{utkozes2.html}{utkozes2.html}}
    \scriptsize
    \lstinputlisting[style=HTML,linerange={6-13},numbers=left,firstnumber=6]{utkozes2.html}
  \end{exampleblock}
  \begin{columns}[T]
    \column{0.6\textwidth}
      \begin{exampleblock}{\textattachfile{utkozes1.css}{utkozes1.css}}
        \scriptsize
        \lstinputlisting[style=HTML,numbers=left,firstnumber=1]{utkozes1.css}
      \end{exampleblock}
    \column{0.3\textwidth}
      \includegraphics[width=.66\textwidth]{utkozes2.png}
  \end{columns} 
\end{frame}
