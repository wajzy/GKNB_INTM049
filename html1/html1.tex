\documentclass[usenames,dvipsnames,aspectratio=169]{beamer}
\usepackage{../common/web}

\title[Web technológiák - HTML]{Web-technológia}
\subtitle{HTML, I. rész}

\begin{document}

%1
\begin{frame}[plain]
  \titlepage
  \logoalul
\end{frame}

\section{Jelölőnyelvek}

\input{jelolonyelvek.tex}

\section{HTML5}

\subsection{Általános tulajdonságok}

\input{altalanosTulajdonsagok.tex}

\subsection{Az első HTML oldal elkészítése}

\input{elsoHTML.tex}

\subsection{Címsorok}

\input{cimsorok.tex}

\subsection{Bekezdések}

\input{bekezdesek.tex}

\subsection{Soron belüli jelölések}

\input{soronBeluli.tex}

\subsection{Idézés, címek, írásirány}

\input{idezesCimek.tex}

\subsection{Képek és hivatkozások}

\input{kepekHivatkozasok.tex}

\subsection{Táblázatok}

%39
\begin{frame}
  Táblázatok kötelező elemei:
  \begin{description}[m]
    \item[\texttt{<table>}] \hfill \\ Maga a táblázat.
    \item[\texttt{<tr>}] (table row) \hfill \\ A táblázat egy sora, a \texttt{<table>} elembe kell beágyazni.
    \item[\texttt{<td>}] (table data) \hfill \\ A sor egy cellája, a \texttt{<tr>} elembe kell beágyazni, vagy
    \item[\texttt{<th>}] (table header) \hfill \\ fejléc cella. (Általában alapértelmezés szerint félkövér, középre zárt megjelnítésű.)
  \end{description}
\end{frame}


\subsection{Listák, felsorolások}

\input{listak.tex}

\subsection{Soron belüli keretek}

%58
\begin{frame}
  \begin{itemize}
    \item Egy HTML oldal megjelenítése egy másik oldalban: \texttt{<iframe>} elemmel
    \item Amelyik böngésző nem támogatja, megjeleníti a nyitó- és záró címkék (tag-ek) közötti szöveget
  \end{itemize}
  Attribútumok:
  \begin{description}[m]
    \item[\texttt{src}] (source) \hfill \\ A keretbe betöltendő dokumentum URL-je
    \item[\texttt{width}] \hfill \\ Keret szélessége képpontban
    \item[\texttt{height}] \hfill \\ A keret magassága képpontban
  \end{description}
\end{frame}

%59
\begin{frame}
  További attribútumok:
  \begin{description}[m]
    \item[\texttt{name}] \hfill \\ Ez azonosítja a keretet, amibe pl. új tartalom tölthető egy \texttt{<a>} elemmel, ha annak \texttt{target} attribútuma a \texttt{name} értékét tartalmazza
    \item[\texttt{srcdoc}] \hfill \\ Megjelenítendő dokumentum HTML kódja (magasabb prioritású, mint \texttt{src}, ha támogatott)
    \item[\texttt{sandbox}] \hfill \\ Megjelenítési környezet korlátozásainak feloldása (\texttt{allow-forms}, \texttt{allow-pointer-lock}, \texttt{allow-popups}, \texttt{allow-same-origin}, \texttt{allow-scripts}, \texttt{allow-top-navigation}) \hiv{\href{https://www.html5rocks.com/en/tutorials/security/sandboxed-iframes/}{részletek}}
  \end{description}
\end{frame}

%60
\begin{frame}
  Megjegyzések
  \begin{itemize}
    \item Az \texttt{srcdoc}-ot csak az \hiv{\href{https://caniuse.com/\#search=iframe\%20srcdoc}{újabb}} böngészők támogatják
    \item \texttt{srcdoc} tartalmazhat \texttt{<iframe>} elemet \texttt{src} attribútummal, így közvetve teljes weboldal is hivatkozható
    \item \texttt{src}+\texttt{sandbox} biztonságos korszerű böngészőkben, de \kiemel{nem biztonságos a \texttt{sandbox}-ot nem támogatókban!}
    \item \texttt{srcdoc}+\texttt{sandbox} biztonságos korszerű böngészőkben, és nem működik (=biztonságos) az elavultakban
  \end{itemize}
\end{frame}


\subsection{A fejrész elemei}

%62
\begin{frame}
  A \texttt{<head>} tartalmazza a HTML oldal metaadatait. HTML5-től elhagyható, de javasolt használni. Beágyazható elemek:
  \begin{description}[m]
    \item[\texttt{<title>}] \hfill \\ A dokumentum címe, kötelező.
    \item[\texttt{<style>}] \hfill \\ CSS stílusok, formázás megadása; ált. jobb külön fájlba helyezni, ld. később 
    \item[\texttt{<base>}] \hfill \\ A relatív URL-ek a \texttt{href} értéke alapján lesznek értelmezve. A \texttt{target} más elemek \texttt{target} attribútumának alapértelmezett értékét adja meg.
  \end{description}
\end{frame}

%63
\begin{frame}
  \begin{description}[m]
    \item[\texttt{<link>}] \hfill \\ Külső erőforrás és a dokumentum kapcsolatát adja meg. Jellemző alkalmazásai:
    \begin{itemize}
      \item Stíluslap meghatározása: \texttt{href}-ben a CSS fájl URL-je, \texttt{rel} (relationship) \emph{stylesheet}, a \texttt{type} \emph{text/css} értékű.
      \item Ikon (favicon = favorite icon) beállítás: \texttt{href}-ben az ikon URL-je, \texttt{rel} \emph{icon}, a \texttt{type} pl. \emph{image/svg+xml} értékű.
    \end{itemize}
    \item[\texttt{<meta>}] \hfill \\ HTTP fejlécek kulcs (\texttt{http-equiv} attribútum) - érték (\texttt{content} attribútum) párok formájában történő megadására. Jellemző kulcsok:
    \begin{itemize}
      \item \emph{content-type}, a MIME típus és karakterkódolás megadására: \texttt{content="text/html; charset=UTF-8"} $\to$ HTML5-től: csak a \texttt{charset="UTF-8"} attribútummal
      \item \emph{refresh}, automatikus újratöltés, pl. percenként: \texttt{content="60"}
    \end{itemize}
  \end{description}
\end{frame}

%64
\begin{frame}
  \begin{description}[m]
    \item[\texttt{<meta>}] \hfill \\ Metaadatok kulcs (\texttt{name} attribútum) - érték (\texttt{content} attribútum) párok formájában történő megadására. Jellemző kulcsok:
    \begin{itemize}
      \item \emph{description}, weboldal általános leírása
      \item \emph{keywords}, kulcsszavak keresőmotoroknak az oldal tartalmához kapcsolódóan
      \item \emph{author}, szerző
      \item \emph{viewport}, nézetablak beállítás, \texttt{content="width=device-width, initial-scale=1.0"}. Probléma: mobil eszközök nagy felbontásuak, de kis méretűek, számítógép-kijelzőre optimalizált oldalak gyenge felhasználói élménnyel használhatók. \emph{width=device-width}: a kép szélessége alkalmazkodik az eszköz szélességéhez. \emph{initial-scale=1.0} nagyítás kezdeti értéke. \hiv{\href{https://www.quirksmode.org/mobile/viewports2.html}{Részletek}}
    \end{itemize}
  \end{description}
\end{frame}

%65
\begin{frame}
  \begin{description}[m]
    \item[\texttt{<script>}] \hfill \\ JavaScript programok megadására; előnyösebb a \texttt{<body>} végébe tenni
    \item[\texttt{<noscript>}] \hfill \\ JavaScript támogatás hiányában a közrezárt szöveget megjeleníti. HTML5-től a \texttt{body}-ba is kerülhet.
  \end{description}
  \vfill
  HTML5-től a \texttt{<html>}, \texttt{<head>} és \texttt{<body>} elemek elhagyhatók, de ezt nem ajánljuk.
\end{frame}

% minta, feladat


\subsection{Oldalak főbb részeinek jelölése}

\input{reszek.tex}

\subsection{Számítógépes kódok}

\input{kod.tex}

\subsection{Entitások}

\input{entitas.tex}

\end{document}
