%124
\begin{frame}
  A HTML5-ben beépülő modulok (pl. Flash) és akár programozás nélkül 
  lehet videót és zenét lejátszani a \texttt{<video>} elemmel (de a böngészők codec 
  támogatása hiányos). Használható formátumok:
  \begin{itemize}
    \item MP4 (\hiv{\href{https://caniuse.com/\#feat=mpeg4}{legjobb böngésző támogatás}})
    \item WebM
    \item Ogg
  \end{itemize}
  Ha biztosra akarunk menni: publikálás több formátumban is.\\
  Formátumok közötti konvertálás: pl. 
  \hiv{\href{http://www.mirovideoconverter.com/}{MiroVideoConverter}}\\
  A lejátszás programozható a \hiv{\href{https://developer.mozilla.org/en-US/docs/Web/API/HTMLMediaElement}{Media API}}-val\\
\end{frame}

%125
\begin{frame}
  Ha a \texttt{<video>} elem nem támogatott egy böngészőben, a 
  címkék közötti szöveg jelenik meg. Opcionális attribútumok:
  \begin{description}[m]
    \item[\texttt{<autoplay>}] \hfill \\ A lejátszás azonnal indul; 
    \kiemel{nem ajánlott}, zavarhatja a felhasználót
    \item[\texttt{<controls>}] \hfill \\ Vezérlő gombokat jelenít meg
    \item[\texttt{<width>}, \texttt{<height>}] \hfill \\ A lejátszó 
    ablak szélessége, magassága képpontokban; \kiemel{ajánlott} megadni
    \item[\texttt{<loop>}] \hfill \\ Végtelenített lejátszás
  \end{description}
\end{frame}

%126
\begin{frame}
  \begin{description}[m]
    \item[\texttt{<muted>}] \hfill \\ Némítás
    \item[\texttt{<poster>}] \hfill \\ Egy kép, amit a betöltés 
    alatt / lejátszás megkezdéséig lát a felhasználó. Érték: URL
    \item[\texttt{<preload>}] \hfill \\ Adatfolyam betöltési módja. 
    Érték: \texttt{auto | metadata | none}.
    \item[\texttt{<src>}] \hfill \\ Videó forrása. \kiemel{Nem 
    ajánlott} a használata, mert csak egyetlen forrás nevezhető meg, 
    amit valószínűleg nem támogat minden böngésző. Érték: URL.
  \end{description}
\end{frame}

%127
\begin{frame}
  \begin{exampleblock}{\textattachfile{video1.html}{video1.html}}
    \footnotesize
    \lstinputlisting[style=HTML,linerange={8-13},numbers=left,firstnumber=8]{video1.html}
  \end{exampleblock}
\end{frame}

%128
\begin{frame}
  Több adatforrás is megadható beágyazott \texttt{<source>} elemekkel, melyek 
  közül a böngésző az első támogatott formátumhoz tartozót 
  fogja választani. Attribútumok:
  \begin{description}[m]
    \item[\texttt{<src>}] \hfill \\ Adatforrás. Érték: URL
    \item[\texttt{<type>}] \hfill \\ A forrás MIME típusa.
  \end{description}
\end{frame}

%129
\begin{frame}
  \begin{exampleblock}{\textattachfile{video2.html}{video2.html}}
    \scriptsize
    \lstinputlisting[style=HTML,linerange={8-22},numbers=left,firstnumber=8]{video2.html}
  \end{exampleblock}
\end{frame}

%130
\begin{frame}
  A videók feliratozhatóak is a \texttt{<track>} elemmel. Felirat 
  formátum: \hiv{\href{https://www.w3.org/TR/webvtt1/}{VTT}}. 
  \hiv{\href{https://www.nikse.dk/SubtitleEdit/Online}
  {Online szerkesztő}}, 
  \hiv{\href{https://subtitletools.com/convert-to-vtt-online}{átalakító}}. 
  Attribútumok:
  \begin{description}[m]
    \item[\texttt{<default>}] \hfill \\ Kijelölhető több feliratsáv 
    közül az alapértelmezett.
    \item[\texttt{<kind>}] \hfill \\ Feliratsáv típusa: 
    \texttt{captions | chapters | descriptions | metadata | 
    subtitles} (ez az alapértelmezés).
    \item[\texttt{<label>}] \hfill \\ Feliratsáv címkéje, pl. a 
    felirat nyelve.
    \item[\texttt{<src>}] \hfill \\ A felirat forrása, kötelező. 
    Érték: URL
    \item[\texttt{<srclang>}] \hfill \\ Felirat nyelvének ISO 639-1 
    kódja, pl. hu.
  \end{description}
\end{frame}

%131
\begin{frame}
  \begin{exampleblock}{\textattachfile{video3.html}{video3.html}}
    \scriptsize
    \lstinputlisting[style=HTML,linerange={8-23},numbers=left,firstnumber=8]{video3.html}
  \end{exampleblock}
\end{frame}
